\section{Versuchsziel}
Im Versuch werden radioaktive Isotope von Kalium und Samarium untersucht:
Ziel ist die Bestimmung der Halbwertszeiten von ${}^{40}_{19}\text{K}$ (0.0118\,\% nat. Häufigkeit)
und von ${}^{147}_{62}\text{Sm}$ (14.87\,\% nat. Häufigkeit).\\
Dazu wird ein Methangaszählrohr verwendet, in dem die Strahlung von Proben aus KCl und Sm$_2$O$_3$ Ionenlawinen
auslöst. Die Ladung wird dann elektronisch registriert, die Ereignisse gezählt und mit dem Computer aufgenommen.\\[\baselineskip]
Die Einstellung der Elektronik und eine erste Charakterisierung des Zählrohrs wird mit einem Uranpräparat
durchgeführt. Anschließend werden Messungen mit verschiedenen Kalium- und Samariumproben gemacht und
die Untergrundstrahlung bestimmt.