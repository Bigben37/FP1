\section{Versuchsdurchführung}

\subsection{Bestimmung der Zählrohrcharakteristik mit \texorpdfstring{${}^{238}\text{U}$}{U-238}}
Vor Beginn der Messung wird der Wert der Schwellenspannung des
Einkanalanalysators auf 0.75 eingestellt, so dass gerade kein Rauschen mehr
sichtbar ist. Ein Uranpräparat wird in das Zählrohr eingelegt und die
Charakteristik des Zählrohrs bestimmt: Für Spannungen zwischen 1000\,V und
4000\,V (Schrittweite 100\,V) werden jeweils 50\,s lang die Zählereignisse
detektiert. Vor jeder Messung wird eine 15\,s lange Pause gemacht, so dass sich
die Spannung stabilisieren kann. Diese Einstellung wird für alle folgenden
Messungen beibehalten.\\
Anschließend wird die gleiche Messung ohne das Uranpräparat erneut durchgeführt,
um die Untergrundstrahlung zu bestimmen.

\subsection{Bestimmung der Halbwertszeit von \texorpdfstring{${}^{147}\text{Sm}$}{Sm-147}}
In diesem Abschnitt wird die energiereiche $\alpha$-Strahlung von Samariumoxid
untersucht. Dazu wird eine mit Sm$_2$O$_3$ gefüllte große Aluminiumschale in das Zählrohr
gestellt und die Zählraten für Spannungen zwischen 1000\,V und 2200\,V
(Schrittweite 100\,V) jeweils 50\,s lang gemessen.
Da durch die kurze Messzeit ein hoher Fehler auf die Einzelmessungen entsteht,
wird für einen Punkt in der Mitte des Plateaus (bei 1600\,V) eine längere Messung durchgeführt.
Ein Schätzwert für die Zählrate liegt mit obiger Messung bereits vor. Daher kann mit 
\autoref{eq:messzeit} die Messdauer berechnet werden, die zu einem relativen Fehler von
3\,\%\footnote{Die im Versuch geforderte Genauigkeit von 2\,\% konnte nicht
erreicht werden, da wegen Ausfall der Methangasversorgung nur ein Versuchstag
für Experimente zur Verfügung stand und die Messzeiten deswegen verkürzt werden
mussten.}
führt.


Die Messung des Punktes bei 1600\,V wird anschließend mit einer mittelgroßen
und einer kleinen Aluminiumschale noch einmal gemacht.
Um die benötigte Messzeit abzuschätzen, wird zuerst eine
50\,s lange Messung durchgeführt. Mit dem ungenauen Schätzwert für die Zählrate
wird dann die Messzeit berechnet,
die für einen relativen Fehler von 3\%\footnotemark[1] bzw. 4.5\,\%\footnotemark[1]
auf die Zählrate notwendig ist.

Anschließend wird bei der selben Spannung mit einer leeren Aluschale 1.5 Stunden lang
der Untergrund gemessen.\footnote{Auch hier konnte aus Zeitmangel
die geforderte Genauigkeit der Untergrundmessung
nicht erreicht werden.}





\subsection{Bestimmung der Halbwertszeit von \texorpdfstring{${}^{40}\text{K}$}{K-40}}
Im letzten Versuchsteil wird die $\beta^-$-Strahlung von Kalium untersucht.
Dazu wird eine Aluminiumschale mit ca. 2\,g KCl befüllt und in das Zählrohr
eingesetzt. Die Zählrohrspannung wird in 100\,V-Schritten von 2500\,V auf
4000\,V gesteigert. Für jede Spannungseinstellung werden 50\,s lang die Zerfälle
gezählt.\\
Dann wird eine längere Messung bei 3200\,V Zählrohrspannung an der
selben Probe durchgeführt. Wie oben lässt sich die
benötigte Messzeit berechnen, um einen relativen statistischen
Fehler von weniger als 2\,\% auf die Zählrate zu erhalten.\\
Anschließend wird schrittweise die Masse des KCl in der Aluminiumschale
verringert. Für jede Masse wird 50\,s lang die Anzahl der Zählereignisse
registriert und dann jeweils die Messdauer bestimmt, die für den Fehler von
2\,\% notwendig ist.\\
Schließlich wird mit dem leeren Alubehälter 10 Stunden lang\footnotemark[2] der Untergrund gemessen.


