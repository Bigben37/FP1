\subsection{Messergebnisse und Auswertung}
Die Messungen wurden mit einem digitalen Oszilloskop durchgeführt.
Wegen der begrenzten Auflösung des A/D-Wandlers sind die Messwerte für
die Zeiten und Spannungen nicht kontinuierlich verteilt, sie können 
nur diskrete Werte annehmen. Der Kanalabstand für die Zeitmesswerte beträgt $\Delta t = 0.02$\,\textmu s,
der der Spannung $\Delta U = 0.1$\,mV. \\
Unter Annahme einer Gleichverteilung der Messwerte innerhalb eines Kanals des A/D-Wandlers folgt
für den Fehler auf die Messwerte
\begin{equation}
  s_t = \frac{\Delta t}{\sqrt{12}} = 6\,\text{ns}, \qquad s_U = \frac{\Delta U}{\sqrt{12}} = 29\,\text{\textmu V}
\end{equation}
Beim Ablesen der Treibspannung $U_T$ wurde eine deutliche Veränderung der gemessenen Treibspannung bei unterschiedlich eingestellten 
Anzeigeoffsets beobachtet. Aus diesem Grund wurde der Fehler der Treibspannung auf $s_{U_T} = 0.4$\,V geschätzt. \\
Der Fehler auf den Abstand $d$ zwischen Lichtleiter und Messspitze wurde auf $s_d = 0.05$\,mm gesetzt.
\subsubsection{Bestimmung des Offsets \texorpdfstring{$x_0$}{x0}}
Der Abstand zwischen Messspitze und Lichtleiter wurde aufgrund des Versuchsaufbaus mit einem Offset $x_0$ gemessen. Der Offset wurde separat 
von jedem Versuchspartner einmal bestimmt. Aus den Messwerten wurde der Mittelwert bestimmt (Fehler der Einzelmessung 
$s_{x_i} = 0.2$\,mm \footnote{Die Schieblehre konnte nicht direkt angelegt werden, um den Versuchsaufbau nicht zu beschädigen. 
Dadurch ist konnte man den Offset nicht so genau bestimmen.}):
\begin{equation}
  x_0 = (1.25 \pm 0.14)\,\text{mm}
\end{equation}
\subsubsection{Variation der Nadelposition \texorpdfstring{$d$}{d}}

\begin{figure}[H]
\begin{center}
  \includegraphics[width=\textwidth]{../img/part2/dist02.pdf}
  \caption{Gaußfit mit linearem Untergrund des Peaks bei $d=8.03$\,mm.}
  \label{img:d:exfit}
\end{center}
\end{figure}
\autoref{img:d:exfit} zeigt einen beispielhaften Fit in dieser Messreihe. Insgesamt wurden so 11 Messungen gefittet. Die Gleichung ergibt sich 
aus \autoref{eq:gauss} und einem linearen Untergrund:
\begin{equation}
  U(t) = a + b \cdot t + \tilde{A} \cdot \frac{1}{\sqrt{2  \pi  \cdot \tilde{\sigma}^2}} \cdot
  e^{-\frac{1}{2} \left( \frac{t - t_{\text{c}}}{\tilde{\sigma}^2} \right)^2}
\end{equation}
Für die weiteren Berechnungen sind für die Amplitude $\tilde{A}$, den zeitlichen Schwerpunkt $t_c$ und die Standardabweichung
$\tilde{\sigma}$ die Fehler aus 
den Fits verwendet worden.
Die untergrundbereinigten Messungen (ohne Fits) sind in \autoref{img:distances} dargestellt.
\begin{figure}[H]
\begin{center}
  \includegraphics[width=\textwidth]{../img/part2/distances.pdf}
  \caption{Zeitlicher Verlauf der Spannungen für verschiedene Abstände $d$ zwischen Messspitze und Lichtleiter.}
  \label{img:distances}
\end{center}
\end{figure}

\paragraph{Bestimmung der Ladungsträgermobilität $\mu$}
Für jeden Gaußfit wird die Entfernung $d$ zwischen Lichtleiter und Nadelspitze über dem Zeitpunkt des Maximumdurchlaufs $t_{\text{c}}$ 
aufgetragen (\autoref{img:dist:fitxc}), da der örtliche Schwerpunkt $x_c(t)$ zum Zeitpunkt $t_c$ genau $d$ entspricht. Mit 
\autoref{eq:gaus:params} erhält man so eine proportionale Abhängigkeit des Abstandes $d$ von dem zeitlichen Schwerpunkt $t_c$ mit 
$\mu \cdot E$ als Proportionalitätskonstante.
\begin{equation}
  d = x_c(t_c) = \mu \cdot E \cdot t_c
\end{equation} 
Der so erhaltene Graph wird mit einer Geraden gefittet.
\begin{equation}
  x_c(t_c) = x_0 + m \cdot t_c
\end{equation}

\begin{figure}[H]
\begin{center}
  \includegraphics[width=\textwidth]{../img/part2/dist_fitXc.pdf}
  \caption{Entfernung Lichtleiter-Nadel in Abhängigkeit des Zeitpunktes des Maximumsdurchlaufs $t_{\text{c}}$
  und linearer Fit zur Bestimmung der Mobilität~$\mu$.} %TODO caption
  \label{img:dist:fitxc}
\end{center}
\end{figure}

Der Betrag des Achsenabschnitts $|x_0| = (1.01 \pm 0.03)$\,mm stimmt innerhalb von zwei Standardabweichungen mit dem oben bestimmten Wert des 
Offsets überein.\\
Die Steigung $m$ ist das Produkt aus Ladungsträgermobilität und angelegtem E-Feld:
\begin{equation}
  \label{eq:efield}
  m = \mu \cdot E = \mu \cdot \frac{U}{d}
\end{equation}
mit $U = (48.8 \pm 0.4)$\,V und $d=30$\,mm. Man erhält mit Gauß'scher Fehlerfortpflanzung für $\mu$:
\begin{equation}
  \mu = (3079 \pm 29)\,\frac{\text{cm}^2}{\text{V} \cdot \text{s}}
\end{equation}
Dieser Wert ist gegenüber dem Literaturwert
\begin{equation}
  \mu^{\text{Lit}} = 3900\,\frac{\text{cm}^2}{\text{V} \cdot \text{s}}
\end{equation}
stark erniedrigt. Dies kann mit Gitterfehlern an der Probenoberfläche erklärt werden, da diese die Ladungsträgermobilität verringern.

\paragraph{Bestimmung der Diffusionskonstanten $D_\text{n}$}
Die aus den Fits erhaltenen Standardabweichungen $\tilde{\sigma}$ werden gemäß \autoref{eq:scale} mit $\mu \cdot E$ multipliziert und 
in \autoref{img:dist:fitsigma} über $t_c$ aufgetragen. Die Beweglichkeit $\mu$ erhält man von oben und das E-Feld wird analog zu \autoref{eq:efield} berechnet.
Der Fit erfolgt mit \autoref{eq:gaus:params} und einem Zeitoffset $t_0$.
\begin{equation}
  \sigma(t) = \sqrt{2 \cdot D_\text{n} \cdot (t-t_0)}
\end{equation}
Der Offset könnte von einer ungenauen Triggerung auf den Laserpuls oder seiner nicht verschwindenden räumlichen Ausdehnung
bei t=0 verursacht werden. \\
Bei dem Fit wurde der letzte Messwert ($t=22$\,\textmu s) nicht berücksichtigt, da er offensichtlich weit neben dem Modell liegt.\\
Für den Fit erhält man $\chi ^2$ = 14.81 bei 8 Freiheitsgraden.
Dies bedeutet, dass eine 6.3-prozentige Wahrscheinlichkeit besteht,
die vorliegenden Messwerte zu erhalten,
wenn das Modell und die Fitparameter korrekt sind und die Messwerte normalverteilt um das Modell sind.
Auf dem 5\%-Signifikanzniveau kann die Nullhypothese (Korrektheit von Modell und Normalverteilung der Messwerte)
also nicht abgelehnt werden.

\begin{figure}[H]
\begin{center}
  \includegraphics[width=\textwidth]{../img/part2/dist_fitSigma.pdf}
  \caption{Standardabweichung der gefitteten Gaußkurven in Abhängigkeit des Zeitpunktes des Maximumsdurchlaufs $t{_\text{c}}$
  und Fit zur Bestimmung der Diffusionskonstanten~$D$.}
  \label{img:dist:fitsigma}
\end{center}
\end{figure}
Man erhält für die Diffusionskonstante:
\begin{equation}
  D_\text{n} = (99.6 \pm 1.1)\,\frac{\text{cm}^2}{\text{s}}
\end{equation}
Sie stimmt innerhalb von zwei Standardabweichungen mit dem Literaturwert überein. 
\begin{equation}
  D_\text{n}^{\text{Lit}} = 101\,\frac{\text{cm}^2}{\text{s}} 
\end{equation}

\paragraph{Bestimmung der mittleren Lebensdauer $\tau_\text{n}$}
Die Amplituden werden wegen \autoref{eq:scale} auch mit dem Faktor $\mu \cdot E$ multipliziert. \\
Für die Bestimmung der mittleren Lebensdauer wird der Zusammenhang aus \autoref{eq:gaus:params} mit einen konstanten Offset $a$ verwendet:
\begin{equation}
  A(t) = C \cdot e^{- \frac{t}{\tau_\text{n}}} + a
\end{equation}
Der Offset gibt einen systematischen Fehler an, der durch einen Teil des Untergrundes, welcher nicht mit der linearen Funktion oben 
berücksichtigt wurde, verursacht werden könnte.
Außerdem würde ein solcher Offset der Messwerte entstehen, wenn man davon ausgeht, dass oberflächliche Ladungsträger
schnell mit der mittleren Lebensdauer $\tau_{\text{n}}$ rekombinieren, während Ladungsträger im Inneren des Materials viel stabiler sind.\\
Der Fehler auf die Amplituden kommt aus den Fits der einzelnen Gauß-Peaks. Jedoch muss noch berücksichtigt werden, dass jeder Peak von einem 
eigenen Laserpuls erzeugt wurde. Der relative Fehler auf die Intensität des Lasers wurde auf $s_\text{Laser, rel} = 5\%$ geschätzt. Dadurch 
lässt sich nun der Fehler auf die Amplituden bestimmen mit 
\begin{equation}
  s_A = \sqrt{s_\text{Fit}^2 + s_\text{Laser, rel}^2}
\end{equation}
\begin{figure}[H]
\begin{center}
  \includegraphics[width=\textwidth]{../img/part2/dist_fitA.pdf}
  \caption{Amplituden der gefitteten Gaußkurven in Abhängigkeit des Zeitpunktes des Maximumsdurchlaufs $t_{\text{c}}$
  und Fit zur Bestimmung der mittleren Lebensdauer~$\tau_{\text{n}}$.}
  \label{img:dist:fita}
\end{center}
\end{figure}
Der Fit liefert für die mittlere Lebensdauer:
\begin{equation}
  \tau_\text{n} = (4.37 \pm 0.17)\,\text{\textmu s}
\end{equation}
Dieser Wert liegt weit unter dem Literaturwert: 
\begin{equation}
  \tau_\text{n}^\text{Lit} = (45 \pm 2)\,\text{\textmu s}
\end{equation}
Die Ursache dafür ist Rekombination der Ladungsträger an oberflächlichen Gitterfehlern,
da die Eindringtiefe des Lasers in Germanium nicht hoch ist.

\subsubsection{Variation der Treibspannung \texorpdfstring{$U_T$}{U\_T}}
Die Auswertung für die verschiedenen Treibspannungen $U_T$ erfolgt ähnlich wie oben. In \autoref{img:volts} sind alle gemessenen Gauß-Kurven 
ohne Untergrund dargestellt.
\begin{figure}[H]
\begin{center}
  \includegraphics[width=\textwidth]{../img/part2/voltages.pdf}
  \caption{Zeitlicher Verlauf der Spannungen am Oszilloskop für verschiedene Treibspannungen $U_T$.}
  \label{img:volts}
\end{center}
\end{figure}

\paragraph{Bestimmung der Ladungsträgermobilität $\mu$} 
Aus \autoref{eq:gaus:params} folgt mit $E = \frac{U_T}{l}$:
\begin{equation}
  \frac{1}{E} = \frac{\mu}{d} \cdot t \quad \Rightarrow \quad \frac{1}{U_T} = \frac{\mu}{l\cdot d} \cdot t
\end{equation}
Mit dieser Geraden und einem Offset $\frac{1}{U_0}$ werden die Daten gefittet (\autoref{img:volt:fitxc}).
\begin{figure}[H]
\begin{center}
  \includegraphics[width=\textwidth]{../img/part2/volt_fitXc.pdf}
  \caption{Fit der reziproken Treibspannung über den durch die Fits erhaltenen zeitlichen Erwartungswerten 
  zur Bestimmung der Ladungsträgermobilität $\mu$.}
  \label{img:volt:fitxc}
\end{center}
\end{figure}
Man erhält für die Steigung:
\begin{equation}
  m = (3260 \pm 54)\,\frac{1}{\text{V}\cdot\text{s}}
\end{equation}
Dadurch lässt sich mit:
\begin{equation}
  m = \frac{\mu}{l \cdot (d + x_0)} \Rightarrow \mu = m \cdot l \cdot (d + x_0)
\end{equation}
die Mobilität bestimmen. $l=3$\,cm ist die Länge der Germaniumprobe, $d=(3.15\pm 0.05)$\,cm der gemessene Abstand und $x_0$ 
der oben (\autoref{img:dist:fitxc}) bestimmte Offset. \\
Man erhält für die Mobilität:
\begin{equation}
  \mu = (4073 \pm 89)\,\frac{\text{cm}^2}{\text{V} \cdot \text{s}}
\end{equation}
Der Literaturwert liegt innerhalb des 2-$\sigma$-Intervalls des gemessenen Wertes.
\begin{equation}
  \mu^{\text{Lit}} = 3900\,\frac{\text{cm}^2}{\text{V} \cdot \text{s}}
\end{equation}

\paragraph{Bestimmung der Diffusionskonstanten $D_\text{n}$}
Wie oben wurden die gemessenen Standardabweichungen $\tilde{\sigma}$ nach \autoref{eq:scale} mit $\mu \cdot E$ multipliziert. Dabei 
wurde für jedes $\tilde{\sigma}$ die bei der Messung verwendete Treibspannung $U_T$ zur Bestimmung des elektrischen Feldes $E$ benutzt. 
Durch den großen Fehler der Treibspannung (verursacht durch die oben beschriebene Fehlfunktion des Oszilloskops) fallen Fehler der einzelnen 
$\sigma$ viel zu groß aus. Dies zeigt sich auch am viel zu kleinen $\chi^2$-Wert des Fits (\autoref{img:volt:fitsigma}).
\begin{figure}[H]
\begin{center}
  \includegraphics[width=\textwidth]{../img/part2/volt_fitSigma.pdf}
  \caption{Standardabweichung der gefitteten Gaußkurven in Abhängigkeit des Zeitpunktes des Maximumsdurchlaufs $t{_\text{c}}$
  und Fit zur Bestimmung der Diffusionskonstanten~$D$.}
  \label{img:volt:fitsigma}
\end{center}
\end{figure}
Man erhält für die Diffusionskonstante:
\begin{equation}
  D_\text{n} = (102 \pm 11)\,\frac{\text{cm}^2}{\text{s}}
\end{equation}
Sie stimmt mit dem Literaturwert überein.
\begin{equation}
  D_\text{n}^{\text{Lit}} = 101\,\frac{\text{cm}^2}{\text{s}} 
\end{equation}

\paragraph{Bestimmung der mittleren Lebensdauer $\tau_\text{n}$}
Hier werden wieder wie bei der Diffusionskonstante die einzelnen Amplituden mit dem jeweiligen Faktor $\mu \cdot E$ nach \autoref{eq:scale} 
multipliziert. Des weiteren wird wie bei der Berechnung von $\tau_\text{n}$ bei verschiedenen Abständen $d$ der Fehler der Laseramplitude 
berücksichtigt. Die Messwerte sind in \autoref{img:volt:fita} dargestellt und mit folgender Gleichung gefittet:
\begin{equation}
  A(t) = C \cdot e^{- \frac{t}{\tau_\text{n}}} + a
\end{equation}
\begin{figure}[H]
\begin{center}
  \includegraphics[width=\textwidth]{../img/part2/volt_fitA.pdf}
  \caption{Amplituden der gefitteten Gaußkurven in Abhängigkeit des Zeitpunktes des Maximumsdurchlaufs $t_{\text{c}}$
  und Fit zur Bestimmung der mittleren Lebensdauer~$\tau_{\text{n}}$.}
  \label{img:volt:fita}
\end{center}
\end{figure}
Man erhält für die mittleren Lebensdauer $\tau_\text{n}$:
\begin{equation}
  \tau_\text{n} = (2.91 \pm 0.16)\,\text{\textmu s}
\end{equation}
Dieser Wert liegt wieder weit unterhalb des Literaturwerts, die Abweichung folgt aus der gleichen Ursache wie oben beschrieben.
\begin{equation}
  \tau_\text{n}^\text{Lit} = (45 \pm 2)\,\text{\textmu s}
\end{equation}