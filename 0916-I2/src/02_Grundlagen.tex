\section{Physical Principles}
\subsection{Born-Oppenheimer approximation}
Since molecules have a much more complicated Hamiltonian than a hydrogen atom, it is not possible to solve the Schrödinger equation analytical. 
A good approximation for this problem is the \emph{Born-Oppenheimer approximation}.  \\
The reasoning is in the following way: The nuclei are much heavier than the electrons of the molecule ($m_{\text{nucleus}} >> m_{\text{electron}}$). 
Thus the electrons live in a much shorter time scale than the nuclei. They react almost immediately to 
changes of the nuclei and are only minimally affected by the proper motion of the nuclei. According to the \emph{Born-Oppenheimer approximation} 
you can write the whole wave function as a product of the wave functions of the nuclei and the electrons:
\begin{equation}
\label{eq:boapprox}
  \Psi_{\text{total}} = \Psi_{\text{electrons}} \cdot \Psi_{\text{nuclei}}
\end{equation}

\subsection{Energy levels of oscillation}
In a two atomic molecule the two nuclei interact with each other in a mutual potential $V$. In analogy to the two-body problem in classical 
mechanics you can reduce the problem to a problem with only one particle with a reduced mass $\mu = \frac{m_1 m_2}{m_1 + m_2}$ and an effective 
potential $V(r)$. The Schrödinger equation is 
\begin{equation}
\label{eq_schroedinger}
  \left( - \frac{\hbar^2}{2 \mu} \frac{\difd^2}{\difd r^2} + V(r) \right) \Psi(r) = E \Psi(r) .
\end{equation}
For small oscillations around the minimum $r_e$ you can perform a series expansion on the potential $V(r)$:
\begin{equation}
\begin{split}
\label{eq_potential_series}
  V(r) = &  V(r_e) + \underbrace{\left. \frac{\partial V(r)}{\partial r} \right|_{r=r_e}}_{=0} (r-r_e)
     + \left. \frac{1}{2} \frac{\partial^2 V(r)}{\partial r^2} \right|_{r=r_e}(r-r_e)^2 \\
  & + \left. \frac{1}{6} \frac{\partial^3 V(r)}{\partial r^3} \right|_{r=r_e}(r-r_e)^3 + \ldots \\
  = & V(r_e) + \frac{1}{2} V''(r_e)(r-r_e)^2 + \frac{1}{6} V'''(r_e)(r_re)^3 + \ldots
\end{split}
\end{equation}
If you approximate the potential only up to second order the potential equals the potential of a harmonic oscillator with known frequencies and 
eigenenergies:
\begin{equation}
\label{eq:ho:freq}
  \omega = \sqrt{\frac{V''(r_e)}{m}}
\end{equation}
\begin{equation}
\label{eq:ho:energy}
  E(\nu) =  \hbar \omega \left( \nu + \frac{1}{2} \right), \qquad \nu = 0, 1, 2, \ldots
\end{equation}
For higher energy levels this approximation is not good enough and you need to consider higher orders of the series expansion. You can solve the
Schrödinger equation with perturbation theory and it yields the eigenenergies of an inharmonic oscillator:
\begin{equation}
\label{eq:iho:energy}
  G(\nu) = \omega_e \left( \nu + \frac{1}{2} \right) - \omega_e x_e \left( \nu + \frac{1}{2} \right)^2 
            + \omega_e y_e \left( \nu + \frac{1}{2} \right)^3 + \ldots, \qquad \nu = 0, 1, 2, \ldots
\end{equation}
The factor $\omega_e$ corresponds to the oscillation frequency of the harmonic oscillator, the factors $\omega_e x_e$, $\omega_e y_e, \ldots$ 
describe the inharmonicity of the potential. \\
The difference between two adjacent eigenenergies is
\begin{equation}
\label{eq:iho:energydiff}
  \Delta G \left( \nu + \frac{1}{2} \right) = G(\nu + 1) - G(\nu) = \omega_e - \omega_e x_e (2\nu + 2) + \omega_e y_e \left( 3\nu^2 + 6 \nu + \frac{13}{4} \right) + \ldots 
\end{equation}
Provided that $\omega_e x_e$ is positive (and it is for the $I_2$-molecule) there is an 
$\nu_{\text{diss}}$ with $\Delta G (\nu_{\text{diss}} + \frac{1}{2}) = 0$. If the energy of the molecule is above this energy level it 
is dissociated, the atoms split into two ions. \\
The dissociation energy from the lowest energy level is
\begin{equation}
\label{eq:dissenergy}
  D_0 = \sum_{\nu=0}^{\nu_{\text{diss}}} \Delta G \left( \nu + \frac{1}{2} \right),
\end{equation}
and from the minimum of the potential
\begin{equation}
\label{eq:dissenergy2}
  D_e = G(0) + D_0
\end{equation} 
Since $\hbar \omega = h f$, $f = \frac{c}{\lambda}$ and $\lambda = \frac{1}{\bar{\nu}}$ the energy of vibrational states often is denoted in 
wavenumbers $\bar{\nu}$ (in multiples of $h c$, because $E = hc\bar{\nu}$) with the dimension $\text{cm}^{-1}$.

\subsection{Morse potential}
A good approximation of the real potential of a two atomic molecule is the \emph{Morse potential}
\begin{equation}
  V(r) = D_e \left( 1 - e^{-a(r-r_e)} \right)^2
\end{equation}
with dissociation energy (from minimum of potential) $D_e$ , a molecule specific constant $a$  and the internuclear distance $r_e$. \\
The Schrödinger equation for this potential yields for the constants in \autoref{eq:iho:energy}:
\begin{equation}
\label{eq:morse_we}
  w_e = a \sqrt{\frac{\hbar D_e}{\pi c \mu}}
\end{equation}
\begin{equation}
\label{eq_morse_wexe}
  w_e x_e = \frac{\hbar a^2}{2 \omega_e x_e}
\end{equation}
The Dissociation energy can be calculated with those two constants:
\begin{equation}
\label{eq:morse_dissenergy}
  D_e = \frac{\omega_e^2}{4 \omega_e x_e}
\end{equation}

\subsection{Absorption of radiation}
-Frank-Condon principle \\
-progression

\subsection{Rotation}
-formula with rotational constant

\subsection{Statistics}
\subsubsection{Weighted mean}
Different values $x_i$ with individual errors $s_i$ are given. The weighted (arithmetic) mean and its error is:
\begin{equation}
\label{eq:meanw}
  \bar{x} = \frac{\sum_i \frac{x_i}{s_i^2}}{\sum_i \frac{1}{s_i^2}}, \qquad s_{\bar{x}}^2 = \frac{1}{\sum_i \frac{1}{s_i^2}}
\end{equation}