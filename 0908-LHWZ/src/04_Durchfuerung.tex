\section{Versuchsdurchführung}

\subsection{Bestimmung der Zählrohrcharakteristik mit ${}^{238}\!\text{U}$}
Vor Beginn der Messung wird der Wert der Schwellenspannung des
Einkanalanalysators auf 0.75 eingestellt, so dass gerade kein Rauschen mehr
sichtbar ist. Ein Uranpräparat wird in das Zählrohr eingelegt und die
Charakteristik des Zählrohrs bestimmt: Für Spannungen zwischen 1000\,V und
4000\,V (Schrittweite 100\,V) werden jeweils 50\,s lang die Zählereignisse
detektiert. Vor jeder Messung wird eine 15\,s lange Pause gemacht, so dass sich
die Spannung stabilisieren kann. Diese Einstellung wird für alle folgenden
Messungen beibehalten.\\
Anschließend wird die gleiche Messung ohne das Uranpräparat erneut durchgeführt,
um die Untergrundstrahlung zu bestimmen.

\subsection{Bestimmung der Halbwertszeit von ${}^{147}\!\text{Sm}$}
In diesem Abschnitt wird die energiereiche $\alpha$-Strahlung von Samariumoxid
untersucht. Dazu wird eine mit SmO gefüllte Aluminiumschale in das Zählrohr
gestellt und die Zählraten für Spannungen zwischen 1000\,V und 2200\,V
(Schrittweite 100\,V) jeweils 50\,s lang gemessen. Um die Zählrate auf XXX
Prozent\footnote{Die im Versuch geforderte Genauigkeit von 2\,\% konnte nicht
erreicht werden, da wegen Ausfall der Methangasversorgung nur ein Versuchstag
für Experimente zur Verfügung stand und die Messzeiten deswegen verkürzt werden
mussten.}
genau zu ermitteln, wird bei 1600\,V eine 1200\,s lange Messung durchgeführt.\\
Mit zwei weiteren Aluminiumschalen mit kleineren Flächen wird ebenfalls die
Aktivität gemessen. Um die benötigte Messzeit abzuschätzen, wird zuerst eine
50\,s lange Messung durchgeführt. Mit dem ungenauen Schätzwert für die Zählrate
wird dann die Messzeit berechnet; sie beträgt für die mittlere Größe 3600\,s
(XXX rel. Fehler\footnotemark[1]) und 2400\,s (3\% rel Fehler\footnotemark[1]).




\subsection{Bestimmung der Halbwertszeit von ${}^{40}\!\text{K}$}
Im letzten Versuchsteil wird die $\beta^-$-Strahlung von Kalium untersucht.
Dazu wird eine Aluminiumschale mit ca. 2\,g KCl befüllt und in das Zählrohr
eingesetzt. Die Zählrohrspannung wird in 100\,V-Schritten von 2500\,V auf
4000\,V gesteigert. Für jede Spannungseinstellung werden 50\,s lang die Zerfälle
gezählt.\\
Anschließend wird eine Aktivitätsmessung bei 3200\,V Zählrohrspannung an der
selben Probe durchgeführt. Ein Schätzwert für die Zählrate bei dieser Spannung
ist aus obiger Messung schon vorhanden. Mit diesem Schätzwert lässt sich die
benötigte Messzeit berechnen [{Formel!!}], um einen relativen statistischen
Fehler von weniger als 2\,\% auf die Aktivität zu erhalten. Sie beträgt
420\,s.
Anschließend wird schrittweise die Masse des KCl in der Aluminiumschale
verringert. Für jede Masse wird 50\,s lang die Anzahl der Zählereignisse
registriert und dann jeweils die Messdauer bestimmt, die für den Fehler von
2\,\% notwendig ist. Die Messdauer nimmt zu bis auf 780\,s.\\
Schließlich wird mit dem leeren Alubehälter der Untergrund gemessen. Um einen
nichtbeitragenden Fehler zu erhalten, wäre eine Messdauer von XXXX\,s nötig [Formel!!!]. Aus Zeitmangel wird
die Dauer der Messung auf 10\,h beschränkt.

äää
