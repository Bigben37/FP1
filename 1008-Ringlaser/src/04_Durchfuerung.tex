\section{Versuchsdurchführung}
Am Aufbau werden zwei verschiedene Messungen durchgeführt:
Zuerst wird bei 30\,ms, 45\,ms und 60\,ms Periodendauer einer Scheibenumdrehung
die Position der Quarzscheibe über den ganzen möglichen Einstellbereich
in 1\,cm-Schritten (bei 45\,ms Periodendauer) und 1.5\,cm-Schritten (30\,ms und 60\,ms) variiert.
Mit dem Oszilloskop wird nach Einstellung der beiden Spiegel und der Blende ein geeignetes Bild aufgenommen und
am Computer der zeitliche Abstand von ca. 5-15 Intensitätsmaxima ausgemessen.\\
Anschließend werden vier Messungen mit Variation der Scheibendrehzahl für
jeweils feste Position der Quarzscheibe durchgeführt; bei
36\,cm, 40\,cm, 53\,cm und 57\,cm (Anzeige der Millimeterschraube).
Die Dauer einer Umdrehung wird dabei in 3\,ms-Schritten von 30\,ms auf 60\,ms erhöht.
Die Auswertung erfolgt wie oben.