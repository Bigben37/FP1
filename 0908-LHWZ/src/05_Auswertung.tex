\section{Messergebnisse und Auswertung}

\subsection{Zählrohrcharakteristik mit \uran}
\label{sub:eval:uran}
Die gemessene Abhängigkeit der Zählrate $n$ von \uran\, ohne Untergrund von der Zählrohrspannung $U$ wird in der Abbildung \ref{img:char:uran} gezeigt.
\begin{figure}[H]
\begin{center}
  \includegraphics[width=15cm]{../img/Uran238_Charakteristik.pdf}
  \caption[Zählrohrcharakteristik mit \uran]{Zählrohrcharakteristik mit \uran und Untergrund}
  \label{img:char:uran}
\end{center}
\end{figure}
Man erkennt gut das $\alpha$-Plateau bis ca. $2000\,V$ und das $\beta$-Plateau von $2800\,V$ bis $3500\,V$, welche durch die 
unterschiedlichen Energien der beiden Strahlungen verursacht werden. Des weiteren wurde die Zählrate des Untergrundes in der gleiche Abbildung 
\ref{img:char:uran} abgebildet. Der Anstieg der Zählrate des Untergrundes, verursacht von der Zählrohrspannung, lässt sich gut erkennen. Allerdings 
sieht man auch die statistischen Schwankungen in der Zählrate des Untergrundes (die Messpunkte mit $n = 0 / s$ lassen sich auf der logarithmischen 
Skala nicht darstellen), da die einzelnen Messpunkte nur in einer geringen Zeitspanne ($t=50s$) aufgenommen wurden. Für eine glattere Kurve wäre 
eine längere Messzeit erforderlich gewesen. \\
Die relativen Fehler auf die gemessenen Zählraten $\hat{n}$ beträgt nach \autoref{eq:counts:errors}
\begin{equation}
  \frac{s_{\hat{n}}}{\hat{n}} = \frac{1}{\sqrt{\hat{n} \cdot t_{\hat{n}}}}
\end{equation}
Von der gemessenen Zählrate $\hat{n}$ zieht man die Untergrundszählrate $u$ ab, um die echte Zählrate $n$ zu berechnen.
\begin{equation}
  n = \hat{n} - u
\end{equation}
Der absolute Fehler von der echten Zählrate lässt sich folgendermaßen bestimmen.
\begin{equation}
  s_n = \sqrt{s_{\hat{n}} + s_u}
\end{equation}
Jedoch kann man die Fehler der Zählrate des Urans wegen der logarithmischen Skala nicht mehr gut erkennen, weshalb sie in der Abbildung 
weggelassen wurden.

\subsection{Bestimmung der Halbwertszeit von \samarium}
\subsubsection{$\alpha$-Plateau von \samarium}
\begin{figure}[H]
\begin{center}
  \includegraphics[width=15cm]{../img/Samarium147_Charakteristik.pdf}
  \caption[$\alpha$-Plateau mit \samarium]{$\alpha$-Plateau von \samarium} %TODO bessere beschreibung
  \label{img:char:samarium}
\end{center}
\end{figure}
%TODO Beschreibung usw.

\subsubsection{Messung der Flächen}
Die Durchmesser der verschiedenen Flächen wurden je $N = 6$ mal mit einer Schiebelehre (Fehler $s_d = 0.005$ cm) gemessen. Der Mittelwert $\bar{d}$ berechnet sich mit
\begin{equation}
  \bar{d} = \frac{1}{n} \sum_{i=1}^{n} d_i
\end{equation}
Nun gibt es zwei Arten von Fehlern für den mittleren Durchmesser, der eine entspricht der Standardabweichung\footnote{Allerdings muss hier mit 
dem Gewichtungsfaktor $\frac{1}{N-1}$ gerechnet werden, da aus dem gleichen Datensatz schon der Mittelwert berechnet wurde und somit ein Freiheitsgrad 
weniger verfügbar ist.} 
und der andere lässt sich aus dem Fehler der Einzelmessung berechnen.
\begin{equation}
  s_{\bar{d}, \text{stat}} = \sqrt{\frac{1}{N-1} \sum_{n=1}^{N} \left( d_i - \bar{d} \right)^2} \qquad \text{und} \qquad 
  s_{\bar{d}} = \frac{s_d}{\sqrt{N}} 
\end{equation}
Wie man in \autoref{tab:data:samarium:diameter} sehen kann, ist der statistische Fehler immer größer oder gleicht dem Fehler, der durch die Einzelmessungen 
verursacht wird. Deshalb haben wir im Folgenden mit dem statistischen Fehler die Fehler der abhängigen Größen berechnet.
\begin{table}[H]
\caption{Mittlere Durchmesser mit verschiedenen Fehlern}
\begin{center}
\begin{tabular}{|c|c|c|}
  \hline
  Durchmesser $\bar{d}$ / cm & $s_{\bar{d}, \text{stat}} / \text{cm}$ & $s_{\bar{d}} / \text{cm}$  \\ \hline 
  0.9983 & 0.0068 & 0.002 \\ \hline
  1.6992 & 0.0058 & 0.002 \\ \hline
  2.8792 & 0.0020 & 0.002 \\ \hline
\end{tabular}
\end{center}
\label{tab:data:samarium:diameter}
\end{table}

\subsubsection{Bestimmung der Halbwertszeit von \samarium ~aus den einzelnen Flächen}
\label{subsub:samarium:halflife:single}
Die Fläche und ihr Fehler berechnen sich nun mit
\begin{equation}
  F = \pi  \cdot \left( \frac{d}{2} \right)^2 , \qquad 
  s_F = F \cdot \sqrt{ \left( \frac{\partial F}{\partial d} \cdot s_d \right)^2 } = \frac{d}{2} \cdot \pi \cdot s_d
\end{equation}
In \autoref{tab:data:samarium:area} sind die verschiedenen Flächen mit zugehörigen Fehlern aufgelistet.
\begin{table}[H]
\caption{Verschiedene Fl"achen f"ur die Samariumessung}
\begin{center}
\begin{tabular}{|c|c|c|c|}
  \hline
  Durchmesser $d$ / cm & $s_d$ / cm & Fl"ache $F / \text{cm}^2$ & $s_F / \text{cm}^2$ \\ \hline 
  0.9983 & 0.0068 & 0.7828 & 0.0107 \\ \hline
  1.6992 & 0.0058 & 2.2676 & 0.0156 \\ \hline
  2.8792 & 0.0020 & 6.5106 & 0.0092 \\ \hline
\end{tabular}
\end{center}
\label{tab:data:samarium:area}
\end{table}

Die Zählrate $n$ hängt nach \autoref{eq:samarium:counts} von der Fläche $F$ folgendermaßen ab:
\begin{equation}
\label{eq:samarium:counts_eval}
  n(F) = \frac{A_V \cdot F \cdot R}{4} \Rightarrow A_V = \frac{4 \cdot n}{F \cdot R}
\end{equation}
Daraus lässt sich die Halbwertszeit mit \autoref{eq:samarium:halflife} bestimmen:
\begin{equation}
  T_{1/2}({}^{147}\text{Sm}) = \ln 2 \cdot \frac{N}{A} = \ln 2 \frac{N}{A_V \cdot F \cdot d} \overset{\text{\eqref{eq:samarium:counts_eval}}}{=} 
  \ln 2 \cdot \frac{N \cdot R}{4 \cdot n \cdot d}
\end{equation}
Die Anzahl $N$ der Kerne in \samarium\, ist
\begin{equation}
  N = 2 \cdot N_{\text{Sm}_2\text{O}_3} \cdot h_{\text{rel}} = \frac{m \cdot N_A}{M_{\text{Sm}_2\text{O}_3}} \cdot h_{\text{rel}}
\end{equation}
mit $m$ Masse der Probe, Avogadrokonstante $N_A$, molarer Masse $M_{\text{Sm}_2\text{O}_3}=348.717$u und natürlicher Häufigkeit $h_{\text{rel}} = 0.1487$ von \samarium. \\
Mit $\rho \cdot F = d \cdot m$ kann man nun die Halbwertszeit von \samarium\, in Abhängigkeit der Fläche bestimmen.
\begin{equation}
  \label{eq:samarium:halflife_eval}
  T_{1/2}({}^{147}Sm) = \frac{\ln 2}{2} \frac{R \cdot \rho \cdot N_A \cdot h_{\text{rel}}}{M_{\text{Sm}_2\text{O}_3}} \frac{F}{n} = const. \cdot \frac{F}{n}
\end{equation}
In \autoref{tab:data:samarium} sind die Messdaten der Zählraten für die verschiedenen Flächen aufgelistet.
\begin{table}[H]
\caption{Z"ahlraten von \samarium~f"ur verschiedene Fl"achen mit Fehlern}
\begin{center}
\begin{tabular}{|c|c|c|c|}
  \hline
  Fl"ache $F / \text{cm}^2$ & $s_F / \text{cm}^2$ & Z"ahlrate $n / (1/s)$ & $s_n / (1/s)$ \\ \hline 
  0.7828 & 0.0107 & 0.139 & 0.010 \\ \hline
  2.2676 & 0.0156 & 0.290 & 0.014 \\ \hline
  6.5106 & 0.0092 & 0.678 & 0.026 \\ \hline
\end{tabular}
\end{center}
\label{tab:data:samarium}
\end{table}

Daraus lassen sich folgende Halbwertszeiten berechnen:
\begin{gather}
  T_{1/2}(F=0.783 \text{ cm}^2) = (0.639 \pm 0.05)\cdot 10^{11} \text{ a} \\
  T_{1/2}(F=2.268 \text{ cm}^2) = (0.888 \pm 0.04)\cdot 10^{11} \text{ a}\\
  T_{1/2}(F=6.5106 \text{ cm}^2) = (1.09 \pm 0.04)\cdot 10^{11} \text{ a}
\end{gather}
Man erkennt, dass die Werte für die Halbwertszeiten mit der Fläche zunehmen, was nicht dem theoretischen Modell entspricht (die Abhängigkeit der 
Fläche wurde schon in der Formel berücksichtigt und es wurden ungefähr gleiche Werte erwartet). Außerdem ist der Unterschied der ersten und letzten 
Halbwertszeit ungefähr $59\%$. Deshalb wird im nächsten Abschnitt die Halbwertszeit mit einer anderen Methode bestimmt, 
um etwaige systematische Fehler herauszufiltern.

\subsubsection{Bestimmung der Halbwertszeit von \samarium ~mit einer Ausgleichsgeraden}
Eine Ausgleichsgerade lässt sich mit nur drei Messpunkten nicht so exakt bestimmen, jedoch wurde wegen der großen Unstimmigkeit der Fehler 
zueinander eine andere Methode zur Bestimmung der Halbwertszeit gesucht.
Aus \autoref{eq:samarium:halflife_eval} folgt
\begin{equation}
   n = \frac{\ln 2}{2} \frac{R \cdot \rho \cdot N_A \cdot h_{\text{rel}}}{M_{\text{Sm}_2\text{O}_3} \cdot T_{1/2}({}^{147}\text{Sm})} \cdot F
\end{equation}
Es wird ein Polynom ersten Grades angesetzt:
\begin{equation}
  n(F) = a + b \cdot F
\end{equation}
wobei man mit der Steigung $b$ die Halbwertszeit berechnen kann
\begin{equation}
  T_{1/2}({}^{147}\text{Sm}) = \frac{\ln 2}{2} \frac{R \cdot \rho \cdot N_A \cdot h_{\text{rel}}}{M_{\text{Sm}_2\text{O}_3} \cdot b}
\end{equation}
Der Fehler folgt aus dem Fehler der Steigung $s_b$
\begin{equation}
  s_{T_{1/2}} = T_{1/2}({}^{147}\text{Sm}) \cdot \frac{s_b}{b}
\end{equation}
Der Achsenabschnitt $a$ sollte im Idealfall verschwinden. \\
Die Kurvenanpassung liefert
\begin{gather}
  a = (0.068  \pm 0.012 ) \cdot \frac{1}{\text{s}} \\
  b = (0.0947 \pm 0.0048) \cdot \frac{1}{\text{s}\cdot \text{cm}^2}
\end{gather}
und ist in \autoref{img:samarium:areafit} graphisch dargestellt. 
\begin{figure}[H]
\begin{center}
  \includegraphics[width=15cm]{../img/Samarium147-Flaechenabhaengigkeit.pdf}
  \caption[Flächenabhängigkeit von \samarium]{Flächenabhängigkeit von \samarium bei 1600V}
  \label{img:samarium:areafit}
\end{center}
\end{figure}
Die Rechnung liefert eine Halbwertszeit von
\begin{equation}
  T_{1/2}({}^{147}\text{Sm}) = (1.20 \pm 0.06) \cdot 10^{11} \text{ a}
\end{equation}

\subsubsection{Diskussion der Ergebnisse}
Der Literaturwert für die Halbwertszeit von \samarium\, ist $\tilde{T}_{1/2}({}^{147}\text{Sm}) = 1.06 \cdot 10^{11} \text{ a}$.
Wie schon in \ref{subsub:samarium:halflife:single} erwähnt, stimmen die Halbwertszeiten für die unterschiedlichen Flächen nicht miteinander 
überein. Der Wert für die größte Fläche $T_{1/2}(F=6.5106 \text{ cm}^2) = (1.09 \pm 0.04)\cdot 10^{11} \text{ a}$ stimmt innerhalb einer 
Standardabweichung mit dem Literaturwert überein, die Werte der anderen Flächen weichen jedoch stark von ihm ab. Wir vermuten einen systematischen 
Fehler (TODO Ursachen) \\%TODO Ursachen
Bei der Bestimmung der Halbwertszeit mit der Ausgleichsgeraden errechnet man einen Wert von 
$T_{1/2}({}^{147}\text{Sm}) = (1.20 \pm 0.06) \cdot 10^{11} \text{ a}$, welcher innerhalb von drei Standardabweichungen mit dem Literaturwert 
übereinstimmt. Der Achsenabschnitt sollte laut theoretischem Modell verschwinden, jedoch beträgt er $a = (0.068  \pm 0.012 ) \cdot \frac{1}{\text{s}}$
Dies könnte auf einen systematischen Fehler hinweisen.
%TODO Chi^2

\subsection{Bestimmung der Halbwertszeit von \kalium}
\subsubsection{$\beta$-Plateau von \kalium}
\begin{figure}[H]
\begin{center}
  \includegraphics[width=15cm]{../img/Kalium40_Charakteristik.pdf}
  \caption[$\beta$-Plateau mit \kalium]{$\beta$-Plateau von \kalium}
  \label{img:char:kalium}
\end{center}
\end{figure}
In \autoref{img:char:kalium} ist das $\beta$-Plateau der Zählrate $n$ von \kalium\, ohne Untergrund gegen die Zählrohrspannun $V$ aufgetragen 
und in Vergleich mit der Zählrate von \uran\, gesetzt worden. Die Fehler wurden, wie in Kapitel \ref{sub:eval:uran} beschrieben, bestimmt.
%TODO Erklärung Peak am Anfang + Beschreibung Kurve

\subsubsection{Massenabhängigkeit der Zählrate} %TODO besserer Titel?
Nach \autoref{eq:kalium:countrate} hängt die Zählrate folgendermaßen von der Masse der Probe ab:
\begin{equation}
  n(m) = \frac{f}{2} \frac{A_s \cdot F \cdot \rho}{\mu} \left( 1 - e^{- \frac{\mu \cdot m}{F \cdot \rho}} \right) = a(1-e^{-b \cdot m})
\end{equation}
mit
\begin{equation}
  a = \frac{f}{2} \frac{A_s \cdot F \cdot \rho}{\mu} \qquad \text{und} \qquad b = \frac{\mu}{F \cdot \rho}
\end{equation}
Die beiden Parameter $a$ und $b$ lassen sich mit einer Kurvenanpassung (siehe \autoref{img:kalium:massdep}) bestimmen. \\
Die Werte dafür sind in \autoref{tab:data:kalium} aufgelistet.
\begin{table}[H]
\caption{Z"ahlraten von \kalium f"ur verschiedene Massen mit Fehlern}
\begin{center}
\begin{tabular}{|c|c|c|c|}
  \hline
  Masse $m / g$ & $s_m / g$  & Z"ahlrate $n / (1/s)$ & $s_n / (1/s)$ \\ \hline 
  2.012 & 0.001 & 5.463 & 0.121 \\ \hline
  2.012 & 0.001 & 5.125 & 0.118 \\ \hline
  1.905 & 0.001 & 5.229 & 0.119 \\ \hline
  1.681 & 0.001 & 5.334 & 0.120 \\ \hline
  1.483 & 0.001 & 4.810 & 0.115 \\ \hline
  1.295 & 0.001 & 4.998 & 0.109 \\ \hline
  1.099 & 0.001 & 4.562 & 0.105 \\ \hline
  0.809 & 0.001 & 4.381 & 0.097 \\ \hline
  0.695 & 0.001 & 3.840 & 0.092 \\ \hline
  0.501 & 0.001 & 3.288 & 0.078 \\ \hline
  0.303 & 0.001 & 2.317 & 0.063 \\ \hline
\end{tabular}
\end{center}
\label{tab:data:kalium}
\end{table}
 
\begin{figure}[H]
\begin{center}
  \includegraphics[width=15cm]{../img/Kalium40_Massenabhaengigkeit.pdf}
  \caption[Massenabhängigkeit der Zählrate von \kalium]{Massenabhängigkeit der Zählrate von \kalium bei 3200V}
  \label{img:kalium:massdep}
\end{center}
\end{figure}
Die Werte lauten:
\begin{gather}
  a = (5.397 \pm 0.067) \cdot \frac{1}{s} \\ %TODO runden ja oder nein?
  b = (1.862 \pm 0.064) \cdot \frac{1}{g} 
\end{gather}
Außerdem ergibt sich ein Korrelationskoeffizient von
\begin{equation}
  \rho = \frac{\cov(a, b)}{\sigma_a \sigma_b} = -0.82
\end{equation}
%TODO Chi^2 / DoF Auswertung

\subsubsection{Berechnung der Halbwertszeit}
Aus dem Produkt der beiden Parameter lässt sich die spezifische Aktivität $A_s = \frac{A}{m}$ bestimmen:
\begin{equation}
  A_s = \frac{2 \cdot a \cdot b}{f} \quad \Rightarrow \quad A = m \cdot \frac{2 \cdot a \cdot b}{f}
\end{equation}
Die Halbwertszeit von Kalium lässt sich nach \eqref{eq:kalium:halflife} mit %TODO refeq
\begin{equation}
  T_{1/2}({}^{40}K) = \frac{\ln 2}{1.12} \frac{N}{A}
\end{equation}
bestimmen. \\
Die Anzahl $N$ der Kaliumkerne in Kaliumchlorid ist:
\begin{equation}
  N = \frac{m \cdot N_A}{M_{KCl}} h_{\text{rel}}
\end{equation}
wobei $N_A$ die Avogadrokonstante, $M_{KCl}=74.5483\,u$ die molare Masse von Kaliumchlorid und $h_{\text{rel}}=0.0118\%$ die relative Anzahl von Kaliumkernen 
in Kaliumchlorid ist. \\
Daraus berechnet sich die Halbwertszeit von Kalium mit:
\begin{equation}
  T_{1/2} \left( {}^{40} K \right)  = \frac{\ln 2}{1.12} \frac{N_A \cdot h_{\text{rel}}}{M_{KCl}} \frac{f}{2} \frac{1}{a \cdot b} = const. \cdot \frac{1}{a \cdot b}
\end{equation}
Der Fehler ergibt sich aus den relativen Fehlern von $a$ und $b$ und der Berücksichtigung des Korrelationskoeffizienten $\rho$:
\begin{equation}
  \sigma_{T_{1/2}} = \sqrt{ \left( \frac{\sigma_a}{a} \right)^2 + \left( \frac{\sigma_b}{b} \right)^2 + 2 \cdot \frac{\sigma_a}{a} \cdot \frac{\sigma_b}{b} \cdot \rho   }
\end{equation}
Man erhält folgendes Ergebnis:
\begin{equation}
  T_{1/2} \left( {}^{40} K \right) = (1.20 \pm 0.03) \cdot 10^9\,a  
\end{equation}

\subsubsection{Diskussion des Ergebnisses}


%Tabellen mit Messergebnissen
%Abbildungen
%Auswertung, in der die Analyse der Messdaten beschrieben wird und die Ergebnisse mit ihren Fehlern zusammengestellt werden.