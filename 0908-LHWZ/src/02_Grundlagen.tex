\section{Physikalische Grundlagen}

\subsection{Radioaktiver Zerfall}
\subsubsection{Das Zerfallsgesetz}
Für die Beschreibung des radioaktiven Zerfalls eines Isotops werden seine mittlere Lebensdauer $\tau$ 
und die Zerfallskonstante $\lambda =\frac{1}{\tau}$ verwendet.
Das Produkt $\lambda \cdot dt$ gibt die Wahrscheinlichkeit dafür an, dass ein Kern in der Zeiteinheit $dt$ zerfällt.
Aus der Halbwertszeit $T_{1/2}$ des Elements wird $\lambda$ so berechnet:
\begin{equation}
	\lambda=\frac{\ln (2)}{T_{1/2}}
\end{equation}
Das Produkt aus Zerfallskonstante und Zahl der vorhandenen Kerne $N(t)$ liefert die
Änderung der Kernanzahl $\frac{-d N(t)}{dt}$ und wird als Aktivität $A(t)$ bezeichnet:
\begin{equation}
	A(t)=\lambda N(t)= -\frac{d N(t)}{dt}
\end{equation}
Als Lösung dieser Differenzialgleichung erhält man das Zerfallsgesetz (mit $N_0 = N(0)$):
\begin{equation}
	N(t)=N_0 e^{-\lambda t}
\end{equation}
Damit erhält man für die Aktivität $A(t)$
\begin{equation}
	A(t)=\lambda N_0 e^{-\lambda t}
\end{equation}
Für lange Halbwertszeiten sind Kernanzahl $N$ und Aktivität $A$ nahezu konstant und es gilt
\begin{equation}
	A=\lambda N
\end{equation}

\subsubsection{$\alpha$-Zerfall}
Der $\alpha$-Zerfall findet nach folgendem Schema statt:
\begin{center}
${}^{A}_{Z}\text{X} \rightarrow {}^{A-4}_{Z-2}\text{Y} + {}^{4}_{2}\text{He}$
\end{center}
Der Mutterkern X mit Protonenzahl Z und Nukleonenzahl A zerfällt unter Aussendung eines Helium-Kerns
in den Tochterkern Y.
Der Zerfall findet statt, wenn der Helium-Kern durch den Coulomb-Wall des Mutterkerns tunnelt.\\
\chemel{Uran}{238} und \chemel{Samarium}{147} sind $\alpha$-Strahler:
\chemel{U}{238} zerfällt zu \chemel{Th}{234}, das ebenfalls nicht stabil ist.
Die Energie des ausgesendeten He-Kerns beträgt ca. 4\,MeV.
\chemel{Sm}{147} zerfällt zum stabilen \chemel{Nd}{143}, die Energie beträgt hier ca. 2\,MeV.

\subsubsection{$\beta$-Zerfall}
Beim $\beta^-$-Zerfall zerfällt ein Neutron im Kern zu einem Proton, einem Elektron und einem
Antielektronenneutrino:
\begin{center}
$\text{n} \rightarrow \text{p}^+ + \text{e}^- +\bar{\nu_e}$ bzw.\\[0.15cm]
${}^{A}_{Z}\text{X} \rightarrow {}^{A}_{Z+1}\text{Y} + \text{e}^- + \bar{\nu_e}$
\end{center}
Da beim Zerfall der Impuls unterschiedlich zwischen Elektron und Antineutrino aufgeteilt werden kann,
besitzt $\beta$-Strahlung ein kontinuierliches Energiespektrum, im Gegensatz zum diskreten Spektrum
bei $\alpha$-Strahlung.\\
Der $\beta^-$-Zerfall tritt bei \chemel{K}{40}-Kernen auf.
Sie zerfallen mit einer Wahrscheinlichkeit von 89.28\,\% zu \chemel{Ca}{40},
die Energie der Elektronen beträgt hier maximal 1.3\,MeV.\\
Auch \chemel{Th}{234}, das beim $\alpha$-Zerfall von \chemel{U}{238} entsteht,
zerfällt durch $\beta^-$-Zerfall zu Protaktinium \chemel{Pa}{234},
mit einer Energie der Elektronen von maximal 0.2\,MeV.

\subsubsection{Elektroneneinfang}
Die andere Zerfallsmöglichkeit von \chemel{K}{40} ist der Einfang eines inneren Schalenelektrons
in den Kern und Umwandlung eines Protons in ein Neutron:
\begin{center}
$\text{p}^+ + \text{e}^- \rightarrow \text{n} + \nu_e$ bzw.\\[0.15cm]
${}^{A}_{Z}\text{X} \rightarrow {}^{A}_{Z-1}\text{Y} + \nu_e$
\end{center}
Für \chemel{K}{40} bedeutet das eine Umwandlung zu \chemel{Ar}{40}.\\
Die Zerfallskonstanten für die beiden Zerfallsprozesse von \chemel{K}{40} addieren sich.
Da nur der $\beta^-$-Zerfall detektiert werden kann, muss dies bei der Bestimmung der
Halbwertszeit berücksichtigt werden.

\subsection{Absorption von radioaktiver Strahlung}

\subsection{Funktionsweise des Zählrohrs}

\subsection{Fehlerrechnung bei Zählraten}

Ist der Zerfälle $N$ pro Zeitintervall $t_n$ bekannt, so
beträgt die Zerfallsrate $n$
\begin{equation}
	n=\frac{N}{t_n}
\end{equation}
Da der radioaktive Zerfall poissonverteilt ist, gilt für den Fehler $s_N$ auf $N$
\begin{equation}
	s_N=\sqrt{N}
\end{equation}
Mit dem Gaußschen Fehlerfortpflanzungsgesetz erhält man für den Fehler der Zerfallsrate
\begin{equation}
	s_{n}=
	\sqrt{\left(\frac{\partial }{\partial N}\frac{N}{t_n}\right)^2 \cdot s_N{}^2}=
	\frac{s_N}{t_n}=
	\frac{\sqrt{N}}{t_n}=
	\frac{\sqrt{n \cdot t_n}}{t_n}=
	\sqrt{\frac{n}{t_n}}
\end{equation}
Für den relativen Fehler $s_{n,\text{rel}}$ erhält man damit
\begin{equation}
	s_{n,\text{rel}}=\frac{s_{n}}{n}=
	\frac{1}{\sqrt{n \cdot t_n}}
\end{equation}
bzw.
\begin{equation}
\label{eq:messzeit}
	t_n=
	\frac{1}{n \cdot s_{n,\text{rel}}{}^2}
\end{equation}

Meist ist es nicht möglich, die Zählrate $n$ direkt zu messen, da zusätzlich noch eine Untergrundstrahlung
vorhanden ist. Die gemessene Zählrate $\hat{n}$ muss also um die Untergrundzählrate $u$ bereinigt werden:
\begin{equation}
   n = \hat{n} - u
\end{equation}
Man erhält dann für den Fehler $s_n$
\begin{equation}
  s_n=\sqrt{s_{\hat{n}}{}^2+s_u{}^2}=\sqrt{\frac{\hat{n}}{t_{\hat{n}}}+\frac{u}{t_u}}
\end{equation}
