\section{Physikalische Grundlagen}

\subsection{Das Bändermodell}

Ähnlich zur Beschreibung der Aufenthaltswahrscheinlichkeit von Elektronen mit Molekülorbitalen
wird bei Festkörpern das Bändermodell verwendet.\\
Für die physikalischen Eigenschaften eines Materials
ist die Besetzung von \emph{Valenzband} und \emph{Leitungsband} entscheidend:
Das Valenzband ist (bei 0\,K) das oberste mit Elektronen besetze Band.
Das Leitungsband liegt energetisch darüber und Elektronen im Leitungsband werden als \emph{quasifrei}
bezeichnet.\\
Die Energiedifferenz zwischen Valenz- und Leitungsband wird \emph{Bandlücke} genannt.
Energiewerte, die in der Bandlücke liegen, sind für Elektronen nicht erlaubt.
Es finden nur Übergänge zwischen den Bändern statt,
bei denen die Elektronen mindestens die Bandlückenenergie aufnehmen oder abgeben.
Ist die Bandlücke klein (weniger als 1\,eV), so wird der Festkörper als \emph{Metall} bezeichnet.
Bei Metallen befinden sich bei Raumtemperatur viele Elektronen im Leitungsband und stehen 
für den Ladungstransport zur Verfügung.
Ist die Bandlücke eines Materials größer als 4\,eV, handelt es sich um einen \emph{Isolator},
bei fast alle Elektronen
fest an die Atomrümpfe gebunden sind und kein Ladungstransport stattfinden kann.
Die Stoffe, deren Bandlücke mittelgroß ist, werden als \emph{Halbleiter} bezeichnet.
Bei ihnen wird die Ladungsträgerdichte sehr stark durch Dotierung und Temperatur beeinflusst.\
Elektronische Übergänge zwischen den beiden Bändern sind \emph{direkt} und \emph{indirekt} möglich.
Bei einer direkten Bandlücke liegt im Energie-Wellenvektor-Diagramm das Minimum des Leitungsbandes
genau über dem Maximum des Valenzbandes.
Beim Übergang bleibt daher die Länge des Wellenvektors konstant und
es findet keine Impulsänderung des Elektrons statt.
Bei einer indirekten Bandlücke liegen Minimum und Maximum nicht übereinander.
Ein Elektronenübergang kann hier nur stattfinden,
wenn ein Phonon (Quasiteilchen einer Gitterschwingung) für den Ausgleich der Impulsbilanz sorgt.\\[0.1cm]
Die Bandlücke beeinflusst nicht nur elektrische, sondern auch optische Eigenschaften des Festkörpers:
Transmission von Strahlung ist nur möglich,
wenn die Energie der Photonen kleiner ist als die Bandlücke.
Bei größerer Energie können Elektronen ins Leitungsband gehoben werden und die Photonen werden absorbiert.


\subsection{Dotierung von Halbleitern}

Dotierung Leitfähigkeit
pn-Übergang->RLZ
Sperrrichtung, Durchlassrichtung
Diodenkennlinie
Halbleiterdetektoren


\subsection{Ladungsträger in Halbleitern}

Die Bewegung der Ladungsträger in einem Halbleiter wird durch zwei Prozesse bestimmt:
\emph{Drift} und \emph{Diffision}.
Der Driftstrom $j_{\text{drift}}$ wird durch ein äußeres elektrisches Feld $E$ verursacht
und beträgt
\begin{equation}
\label{}
j_{\text{drift}}=\mu_{\text{n}} \cdot n \cdot E
\end{equation}
Die Mobilität der Elektronen (Verhältnis von Driftgeschwindigkeit zum angelegtem Feld) ist $\mu_{\text{n}}$,
die Ladungsträgerdichte wird als $n$ bezeichnet.
Eine räumliche Inhomogenität von $n$ führt zu einem Diffusionsstrom $j_{\text{diff}}$:
\begin{equation}
\label{}
j_{\text{diff}}=D_{\text{n}} \cdot \frac{\partial n}{\partial x}
\end{equation}
$D_{\text{n}}$ ist die materialspezifische Diffusionskonstante.\
Wegen Ladungserhaltung muss die Ladungsträgerdichte $n(t)$ die \emph{Kontinuitätsgleichung} erfüllen:
Die zeitliche Änderung der Elektronen im Leitungsband wird durch die Zahl der ins Valenzband
zurückfallenden Elektronen und die Divergenz des Stromes $j$ bestimmt:
\begin{equation}
\label{}
\frac{\partial n}{\partial t}= -\frac{n-n_0}{\tau_{\text{n}}}-\frac{\partial j}{\partial x}
\end{equation}
Die Ladungsträgerdichte im Gleichgewicht ist $n_0$, die mittlere Lebensdauer der Elektronen
im Leitungsband $\tau_{\text{n}}$.\\


DGLs lösen für Gaußkurve



TODO:Quelle: engl wiki, Haynes–Shockley experiment



\subsection{Lock-in-Verstärker}


Bilder mit Mathematica + Formeln
