\section{Physikalische Grundlagen}
\subsection{Radioaktiver Zerfall und Strahlung}

Der \textalpha- und \textbeta-Zerfall der verwendeten Proben führt dazu,
dass sich nach dem Zerfall die Kerne in angeregten Zuständen befinden.
Diese Anregungsenergie wird dann als \textgamma-Photon abgegeben.\\

Der \textalpha-Zerfall, der fünf mal in der Zerfallsreihe von ${}^{228}$Th auftritt,
findet nach folgendem Schema statt:
\begin{center}
${}^{A}_{Z}\text{X} \rightarrow {}^{A-4}_{Z-2}\text{Y} + {}^{4}_{2}\text{He}$
\end{center}
Der Mutterkern X mit Protonenzahl Z und Nukleonenzahl A zerfällt unter Aussendung eines Helium-Kerns
in den Tochterkern Y.\\

Beim \textbeta$^-$-Zerfall zerfällt ein Neutron im Kern zu einem Proton, einem Elektron und einem
Antielektronenneutrino:
\begin{center}
$\text{n} \rightarrow \text{p}^+ + \text{e}^- +\bar{\nu_{\text{e}}}$ bzw.\\[0.15cm]
${}^{A}_{Z}\text{X} \rightarrow {}^{A}_{Z+1}\text{Y} + \text{e}^- + \bar{\nu_{\text{e}}}$
\end{center}

Der \textbeta$^-$-Zerfall tritt zwei mal in der Zerfallsreihe von ${}^{228}$Th auf.
Auch ${}^{60}$Co und ${}^{152}$Eu sind \textbeta$^-$-Strahler.

${}^{22}$Na ist ein \textbeta$^+$-Strahler:
Hier zerfällt ein Proton im Kern zu einem Neutron, einem Positron und einem
Elektronenneutrino:
\begin{center}
$\text{p}^+ \rightarrow \text{n} + \text{e}^+ +\nu_{\text{e}}$ bzw.\\[0.15cm]
${}^{A}_{Z}\text{X} \rightarrow {}^{A}_{Z-1}\text{Y} + \text{e}^+ + \nu_{\text{e}}$
\end{center}

Das Positron kann in Materie für nur sehr kurze Zeit existieren und zerstrahlt mit einem Elektron
in zwei Photonen mit je 511\,keV Energie, die in einem Winkel von 180$^\circ$ emittiert werden.

Die emittierten \textgamma-Photonen können auf verschiedene Arten mit Materie wechselwirken:
Beim \emph{Photoeffekt} absorbieren gebundene Elektronen das Photon und verlassen mit hoher Geschwindigkeit das Atom.
Beim \emph{Comptoneffekt} findet elastische Streuung von Photonen an freien Elektronen statt.
Ein Teil der Photonenenergie wird so auf das Elektron übertragen.
Eine weitere Möglichkeit ist die \emph{Paarbildung}, bei der im Feld eines Atomkerns aus einem energiereichen
Photon ein Elektron-Positron-Paar entsteht.



\subsection{Szintillatoren}




\subsection{Photomultiplier}