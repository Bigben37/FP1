\subsection{Messergebnisse und Auswertung}
\subsubsection{Bestimmung des Offsets \texorpdfstring{$x_0$}{x0}}
Der Abstand zwischen Messspitze und Lichtleiter wurde aufgrund des Versuchsaufbaus mit einem Offset $x_0$ gemessen. Der Offset wurde separat 
von jedem Versuchspartner einmal bestimmt. Aus den Messwerten wurde der Mittelwert bestimmt (Fehler der Einzelmessung $s_{x_i} = 0.2$\,mm):
\begin{equation}
  x_0 = (1.25 \pm 0.14)\,\text{mm}
\end{equation}
\subsubsection{Variation der Nadelposition \texorpdfstring{$d$}{d}}

\begin{figure}[H]
\begin{center}
  \includegraphics[width=\textwidth]{../img/part2/dist02.pdf}
  \caption{Gaußfit mit linearem Untergrund des Peaks bei $d=8.03$\,mm.}
  \label{img:d:exfit}
\end{center}
\end{figure}
\autoref{img:d:exfit} zeigt einen beispielhaften Fit in dieser Messreihe. Insgesamt wurden so 11 Messungen gefittet. Die Gleichung ergibt sich 
aus \autoref{eq:gauss} und einem linearen Untergrund:
\begin{equation}
  U(t) = a + b \cdot t + A \cdot \frac{1}{\sqrt{2  \pi  \cdot \sigma^2}} \cdot
  e^{-\frac{1}{2} \left( \frac{t - t_{\text{c}}}{\sigma^2} \right)^2}
\end{equation}
Die untergrundbereinigten Messungen (ohne Fits) sind in \autoref{img:distances} dargestellt.
\begin{figure}[H]
\begin{center}
  \includegraphics[width=\textwidth]{../img/part2/distances.pdf}
  \caption{Zeitlicher Verlauf der Spannungen für verschiedene Abstände $d$ zwischen Messspitze und Lichtleiter.}
  \label{img:distances}
\end{center}
\end{figure}

\paragraph{Bestimmung der Ladungsträgermobilität $\mu$}
Für jeden Gaußfit wird die Entfernung $d$ zwischen Lichtleiter und Nadelspitze über dem Zeitpunkt des Maximumdurchlaufs $t_{\text{c}}$ 
aufgetragen (\autoref{img:dist:fitxc}). Der so erhaltene Graph wird mit einer Geraden gefittet.

\begin{figure}[H]
\begin{center}
  \includegraphics[width=\textwidth]{../img/part2/dist_fitXc.pdf}
  \caption{Ausgleichsgerade} %TODO caption
  \label{img:dist:fitxc}
\end{center}
\end{figure}

Der Betrag des Achsenabschnitts $|x_0| = (1.01 \pm 0.03)$\,mm stimmt innerhalb von zwei Standardabweichungen mit dem oben bestimmten Wert des 
Offsets überein.\\
Die Steigung $m$ ist das Produkt aus Ladungsträgermobilität und angelegtem E-Feld 
(nach \autoref{eq:gaus:params}):
\begin{equation}
  \label{eq:efield}
  m = \mu \cdot E = \mu \cdot \frac{U}{d}
\end{equation}
mit $U = (48.8 \pm 0.4)$\,V und $d=30$\,mm. Man erhält mit Gauß'scher Fehlerfortpflanzung für $\mu$
\begin{equation}
  \mu = (3079 \pm 29)\,\frac{\text{cm}^2}{\text{V} \cdot \text{s}}
\end{equation}
Dieser Wert ist gegenüber dem Literaturwert
\begin{equation}
  \mu^{\text{Lit}} = 3900\,\frac{\text{cm}^2}{\text{V} \cdot \text{s}}
\end{equation}
stark erniedrigt. Dies kann mit Gitterfehlern an der Probenoberfläche erklärt werden, da diese die Ladungsträgermobilität verringern.

\paragraph{Bestimmung der Diffusionskonstanten $D_\text{n}$}
Die aus den Fits erhaltenen Standardabweichungen $\tilde{\sigma}$ werden gemäß \autoref{eq:gauss} mit $\mu \cdot E$ multipliziert und 
in \autoref{img:dist:fitsigma} über $t_c$ aufgetragen. Die Beweglichkeit $\mu$ erhält man von oben und das E-Feld wird analog zu \autoref{eq:efield} berechnet.
Der Fit erfolgt mit \autoref{eq:gaus:params} und einem Zeitoffset $t_0$.
\begin{equation}
  \sigma(t) = \sqrt{2 \cdot D_\text{n} \cdot (t-t_0)}
\end{equation}
Der Offset könnte von einer ungenauen Triggerung auf den Laserpuls oder seine Ausdehnung verursacht werden. \\
Bei dem Fit wurde der letzte Messwert ($t=22$\,\textmu s) nicht berücksichtigt, da er offensichtlich weit neben dem Modell liegt.

\begin{figure}[H]
\begin{center}
  \includegraphics[width=\textwidth]{../img/part2/dist_fitSigma.pdf}
  \caption{caption}
  \label{img:dist:fitsigma}
\end{center}
\end{figure}
Man erhält für die Diffusionskonstante:
\begin{equation}
  D_\text{n} = (99.6 \pm 1.1)\,\frac{\text{cm}^2}{\text{s}}
\end{equation}
Sie stimmt innerhalb von zwei Standardabweichungen mit dem Literaturwert überein. 
\begin{equation}
  D_\text{n}^{\text{Lit}} = 101\,\frac{\text{cm}^2}{\text{s}} 
\end{equation}

\paragraph{Bestimmung der mittleren Lebensdauer $\tau_\text{n}$}
Für die Bestimmung der mittleren Lebensdauer wird der Zusammenhang aus \autoref{eq:gaus:params} mit einen konstanten Offset verwendet:
\begin{equation}
  A(t) = C \cdot e^{- \frac{t}{\tau_\text{n}}} + a
\end{equation}
Der Offset gibt einen systematischen Fehler an, der durch einen Teil des Untergrundes, welcher nicht mit der linearen Funktion oben 
berücksichtigt wurde, verursacht werden könnte.

\begin{figure}[H]
\begin{center}
  \includegraphics[width=\textwidth]{../img/part2/dist_fitA.pdf}
  \caption{caption}
  \label{img:dist:fita}
\end{center}
\end{figure}
Der Fit liefert für die mittlere Lebensdauer:
\begin{equation}
  \tau_\text{n} = (4.37 \pm 0.17)\,\text{\textmu s}
\end{equation}
Dieser Wert liegt weit unter dem Literaturwert. Die Ursache dafür ist Rekombination der Ladungsträger an oberflächlichen Gitterfehlern.
\begin{equation}
  \tau_\text{n}^\text{Lit} = (45 \pm 2)\,\text{\textmu s}
\end{equation}


\subsubsection{Variation der Treibspannung \texorpdfstring{$U_T$}{U\_T}}

\begin{figure}[H]
\begin{center}
  \includegraphics[width=\textwidth]{../img/part2/voltages.pdf}
  \caption{Zeitlicher Verlauf der Spannungen für verschiedene Treibspannungen $U_T$.}
  \label{img:volts}
\end{center}
\end{figure}

\paragraph{Bestimmung der Ladungsträgermobilität $\mu$} 
Platzhalter

\begin{figure}[H]
\begin{center}
  \includegraphics[width=\textwidth]{../img/part2/volt_fitXc.pdf}
  \caption{caption}
  \label{img:volt:fitxc}
\end{center}
\end{figure}

\paragraph{Bestimmung der Diffusionskonstanten $D_\text{n}$}
Platzhalter

\begin{figure}[H]
\begin{center}
  \includegraphics[width=\textwidth]{../img/part2/volt_fitSigma.pdf}
  \caption{caption}
  \label{img:volt:fitsigma}
\end{center}
\end{figure}

\paragraph{Bestimmung der mittleren Lebensdauer $\tau_\text{n}$}
Platzhalter

\begin{figure}[H]
\begin{center}
  \includegraphics[width=\textwidth]{../img/part2/volt_fitA.pdf}
  \caption{caption}
  \label{img:volt:fita}
\end{center}
\end{figure}