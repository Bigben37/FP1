\section{Versuchsziel}
Im Halbleiter-Versuch werden in drei verschieden Versuchsteilen unterschiedliche
halbleiterphysikalische Effekte untersucht:
Mit einer Transmissions- und Absorptionsmessung werden die \emph{Bandlücken} von Germanium und Silizium bestimmt.
Beim Haynes-Shockley-Experiment erhält man Informationen über die \emph{Mobilität}, \emph{Diffusionskonstante} und
\emph{mittlere Lebensdauer} der Elektronen im Leitungsband von Germanium.
Außerdem wird mit dotierten Halbleitern das \emph{Energiespektrum} von radioaktiven Proben aufgenommen.