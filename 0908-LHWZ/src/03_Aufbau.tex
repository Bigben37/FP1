\section{Versuchsaufbau}

Das Methandurchflusszählrohr, das im Versuch verwendet wird, ist über eine Auswerteelektronik mit
einem Computer verbunden: Die Stromimpulse, die durch Ionisationen im Zählrohr erzeugt werden, 
werden erst in einen Vorverstärker und dann in einen weiteren Verstärker geleitet.
Daraufhin gelangt das verstärkte Signal in einen Einkanalanalysator, der das Signal so modifiziert,
dass anschließend die Impulse gezählt werden können. Die Auswertung erfolgt mit einem
\textit{Labview}-Programm.

Das Zählrohr ist an eine Hochspannungsquelle angeschlossen und wird mit Methan aus einer Gasflasche gespült.
Die Hochspannungsquelle wird mit \textit{Labview} angesteuert, die Regelung des Gasdurchflusses erfolgt mit
mechanischen Druckreglern.