\subsection{Versuchsdurchführung}

Am Aufbau für den Faraday-Effekt wird nur eine Messreihe durchgeführt:
Die Drehung der Polarisationsrichtung wird in Abhängigkeit des Spulenstroms $I$
(Variation von -5\,A bis 5\,A in Schritten von 0.5\,A) untersucht.
Jeder Messpunkt wird insgesamt vier mal aufgenommen.
Um eine Drift der Anlage auszuschließen,
werden zuerst die ganzzahligen Messwerte aufgenommen und anschließend die
Zwischenwerte.
Nach der Einstellung des Stroms für jeden Messpunkt wird der Analysator
des Halbschattenpolarimeters so eingestellt, dass die äußeren Strahlteile genauso hell sind wie der innere
Strahlteil.
Es gibt dafür eine helle und eine dunkle Einstellung.
Wegen der logarithmischen Empfindlichkeitskurve des menschlichen Auges wird die dunklere Einstellung
für die Messung verwendet, da daraus eine höhere Messgenauigkeit folgt.
