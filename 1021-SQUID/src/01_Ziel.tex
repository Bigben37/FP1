\section{Versuchsziel}
Im Versuch wird ein Messprinzip angewendet,
mit dem sehr kleine Änderungen der Magnetfeldstärke detektiert werden können:
Ein \emph{Superconducting Quantum Interference Device} verliert im Magnetfeld seine Supraleitfähigkeit,
wenn sich die Magnetfeldstärke um mehr als den Betrag eines Flussquants ändert.
Dies führt zur Dämpfung eines elektrischen Wechselfelds, die detektiert werden kann.
Das Messprinzip wird verwendet, um die winkelabhängige Stärke des Magnetfelds einer \emph{Leiterschleife},
von \emph{Eisen-, Gold-} und \emph{Magnetspänen} und die eines kleinen \emph{Stabmagneten} zu vermessen.