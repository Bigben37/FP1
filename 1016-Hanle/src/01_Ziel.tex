\section{Versuchsziel}
Das Ziel des Versuches ist die Bestimmung der mittleren Lebensdauer des Übergangs
6p6s~$\rightarrow$~6s$^2$ ($^3$P$_1 \rightarrow \, ^1$S$_0$) von Quecksilber.
Die Lebensdauer wird unter Ausnutzung des \emph{Hanle-Effekts} bestimmt:
Werden Atome von linear polarisiertem Licht angeregt,
so ist die Richtung ihres Dipolmoments und damit auch ihres magnetischen Moments festgelegt.
Die Winkelverteilung, unter der beim Zurückfallen in den Grundzustand ein Photon emittiert wird,
wird durch die Abstrahlcharakteristik eines Hertzschen Dipols beschrieben.
Parallel zur Polarisationsrichtung des Lichts findet also keine Emission statt.
Bei Anwesenheit eines Magnetfelds gilt dies allerdings nicht:
Der Hanle-Effekt beschreibt die Präzession des magnetischen Moments in einem
Magnetfeld senkrecht zur Strahlrichtung des einfallenden Lichts.
Dadurch wird auch das Dipolmoment gedreht und damit die Abstrahlcharakteristik der Atome.
Die Messung der Abstrahlcharakteristik bei verschiedenen Magnetfeldstärken ermöglicht die
Berechnung der mittleren Lebensdauer des angeregten Zustandes.


