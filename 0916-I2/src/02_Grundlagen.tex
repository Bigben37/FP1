\section{Physical Principles}
The formulas and explanations in this section are based on \cite{bitsch} (p. 6 - 20) 
\subsection{Born-Oppenheimer approximation}
Since molecules have a much more complicated Hamiltonian than a hydrogen atom, it is not possible to solve the Schrödinger equation analytical. 
But there are ways to solve the Schrödinger equation for this problem anyway, if you use approximations for the wave function.
A good approximation for molecules is the \emph{Born-Oppenheimer approximation}.  \\
The reasoning is in the following way: The nuclei are much heavier than the electrons of the molecule ($m_{\text{nucleus}} >> m_{\text{electron}}$). 
Thus the electrons have a much shorter time scale in which they react to changes of the potential of the molecule than the nuclei. Also they are 
only minimally affected by the proper motion of the nuclei. According to the \emph{Born-Oppenheimer approximation} 
you can write the whole wave function as a product of the wave functions of the nuclei and the electrons:
\begin{equation}
\label{eq:boapprox}
  \Psi_{\text{total}} = \Psi_{\text{electrons}} \cdot \Psi_{\text{nuclei}}
\end{equation}

\subsection{Energy levels of vibration}
In a two atomic molecule the two nuclei interact with each other in a mutual potential, which is a function of the internuclear distance. 
In analogy to the two-body problem in classical mechanics you can reduce the problem to a problem with only one particle with a 
reduced mass $\mu = \frac{m_1 m_2}{m_1 + m_2}$ and an effective potential $V(r)$. The Schrödinger equation is 
\begin{equation}
\label{eq_schroedinger}
  \left( - \frac{\hbar^2}{2 \mu} \frac{\difd^2}{\difd r^2} + V(r) \right) \Psi(r) = E \Psi(r) .
\end{equation}
For small vibrations around the minimum of the potential ($r = r_e$), the two nuclei are in equilibrium and you can perform a 
series expansion on the potential $V(r)$:
\begin{equation}
\begin{split}
\label{eq_potential_series}
  V(r) = \, &  V(r_e) + \underbrace{\left. \frac{\partial V(r)}{\partial r} \right|_{r=r_e}}_{=0} (r-r_e)
     + \left. \frac{1}{2} \frac{\partial^2 V(r)}{\partial r^2} \right|_{r=r_e}(r-r_e)^2 \\
  & + \left. \frac{1}{6} \frac{\partial^3 V(r)}{\partial r^3} \right|_{r=r_e}(r-r_e)^3 + \ldots \\
  =: & V(r_e) + \frac{1}{2} V''(r_e)(r-r_e)^2 + \frac{1}{6} V'''(r_e)(r_re)^3 + \ldots
\end{split}
\end{equation}
If you approximate the potential only up to second order the potential equals the potential of a harmonic oscillator with known frequency and 
eigenenergies:
\begin{equation}
\label{eq:ho:freq}
  \omega = \sqrt{\frac{V''(r_e)}{m}}
\end{equation}
\begin{equation}
\label{eq:ho:energy}
  E(\nu) =  \hbar \omega \left( \nu + \frac{1}{2} \right), \qquad \nu = 0, 1, 2, \ldots
\end{equation}
For higher energy levels this approximation is not good enough, since the potential is not symmetric around $r_e$, 
and you need to consider higher orders of the series expansion. You can solve the
Schrödinger equation with perturbation theory and it yields the eigenenergies of an inharmonic oscillator:
\begin{equation}
\label{eq:iho:energy}
  G(\nu) = \omega_e \left( \nu + \frac{1}{2} \right) - \omega_e x_e \left( \nu + \frac{1}{2} \right)^2 
            + \omega_e y_e \left( \nu + \frac{1}{2} \right)^3 + \ldots, \qquad \nu = 0, 1, 2, \ldots
\end{equation}
The factor $\omega_e$ corresponds to the oscillation frequency of the harmonic oscillator, the factors $\omega_e x_e$, $\omega_e y_e, \ldots$ 
describe the inharmonicity of the potential. The order of magnitudes decrease with higher powers of $\nu$
\begin{equation}
  \omega_e >> \omega_e x_e >> \omega_e y_e >> \ldots
\end{equation}
The difference between two adjacent eigenenergies is
\begin{equation}
\label{eq:iho:energydiff}
  \Delta G \left( \nu + \frac{1}{2} \right) = G(\nu + 1) - G(\nu) = \omega_e - \omega_e x_e (2\nu + 2) + \omega_e y_e \left( 3\nu^2 + 6 \nu + \frac{13}{4} \right) + \ldots 
\end{equation}
Provided that $\omega_e x_e$ is positive (and it is for the $I_2$-molecule) there is a 
$\nu_{\text{diss}}$ with ${\Delta G (\nu_{\text{diss}} + \frac{1}{2}) = 0}$. If the energy of the molecule is above this energy level it 
is in a dissociated state, the atoms split into two ions. \\
The dissociation energy from the lowest energy level is
\begin{equation}
\label{eq:dissenergy}
  D_0 = \sum_{\nu=0}^{\nu_{\text{diss}}} \Delta G \left( \nu + \frac{1}{2} \right),
\end{equation}
and from the minimum of the potential
\begin{equation}
  \label{eq:dissenergy2}
  D_e = G(0) + D_0
\end{equation} 
Since $\hbar \omega = h f$, $f = \frac{c}{\lambda}$ and $\lambda = \frac{1}{\bar{\nu}}$ the energy of vibrational states and dissociation often 
is denoted in wavenumbers $\bar{\nu}$ 
(in multiples of $h c$, because $E = hc\bar{\nu}$) with the dimension $\text{cm}^{-1}$.

\subsection{Morse potential}
A good approximation of the real potential of a two atomic molecule is the \emph{Morse potential}
\begin{equation}
\label{eq:morse}
  V(r) = D_e \left( 1 - e^{-a(r-r_e)} \right)^2
\end{equation}
with dissociation energy (from the minimum of the potential) $D_e$	, a molecule specific constant $a$  and the internuclear distance $r_e$. \\
The Schrödinger equation for this potential yields for the constants in \autoref{eq:iho:energy}:
\begin{equation}
\label{eq:morse_we}
  w_e = a \sqrt{\frac{\hbar D_e}{\pi c \mu}}
\end{equation}
\begin{equation}
\label{eq:morse_wexe}
  w_e x_e = \frac{\hbar a^2}{4 \pi c \mu}
\end{equation}
The dissociation energy can be calculated with those two constants:
\begin{equation}
\label{eq:morse_dissenergy}
  D_e = \frac{\omega_e^2}{4 \omega_e x_e}
\end{equation}

\subsection{Absorption of radiation}
In the beginning the molecule is in the electronic ground state with vibration quantum number $\nu''$. If now light with a continuous spectrum is 
irradiated, a electron can be lifted into an excited state with vibration quantum number $\nu'$. Since the spectrum of the used light is continuous, there is not only one 
absorption line but many. The lines in their entirety are called a \emph{bond}, where every transition $\nu'' \leftrightarrow \nu'$ corresponds to 
a definite wavelength. All transitions from the same ground state level $\nu''$ are summarized as a \emph{progression}.\\
The probability $\alpha$, which is in accordance with the intensity and can be also found under the name of \emph{Franck-Condon factor}, 
of a transition is calculated via the \emph{Franck-Condon principle}:
\begin{equation}
  \alpha = | \int \Psi_{\nu'}(R) \Psi_{\nu''}(R) \difd V |^2
\end{equation}
with wave functions of the electronic ground state $\Psi_{\nu''}$ and the excited state $\Psi_{\nu'}$.
This integral is also called an overlap integral, because it describes the concentration of the two states in the same space.

\subsection{Allowed transitions}
Basically there is an infinite number of possible transitions into higher states, but not all are allowed. The theoretical foundation is provided 
by \emph{Fermi's golden rule}, which leads to specific transition rules depending on which multipole for approximation is used. In this case the consideration of the 
electric dipole is enough and leads to following rules:
\begin{itemize}
  \item $g \leftrightarrow u, g \nleftrightarrow g, u \nleftrightarrow u$
  \item $\Delta M_j = 0, \pm 1$
  \item $\Delta \Lambda = 0, \pm 1$
  \item $\Delta S = 0$
\end{itemize}
$M_j$ is the magnetic quantum number, $\Lambda$ the total angular momentum and $S$ the total spin of the molecule. The parity is described with
$g$ (even = ``gerade'') and $u$ (odd = ``ungerade''). The ground state of $I_2$ is ${}^{1}\Sigma_{0g}^+$. 
If you use the transition rules onto this state, there are only two excited states which are 
allowed: ${}^3 \Pi_{0u}^-$ and ${}^3 \Pi_{2u}$. But the transition to the second excited state is very weak so that only the transition to the 
first one can be measured in this experiment.


\subsection{Rotation}
While the molecule is vibrating it can also rotate around its center of mass. The quantum-mechanical consideration of the angular momentum 
operator $\hat{L}^2$ yields a correlation between the rotational constant $B_e$
\footnote{With this rotational constant you can write the eigenenergies $\hat{L}^2$ as $E_l = B h c l (l+1)$, where l are the eigenvalues of $\hat{L}^2$. 
But since this experiment does not include the experimental measurement of this constant, the theory has been kept to minimum.}, 
the reduced mass $\mu$ and the equilibrium distance $r_e$ of the two nuclei:
\begin{equation}
  \label{eq:rotconst}
  B_e = \frac{\hbar}{4 \pi \mu r_e^2}
\end{equation}

\subsection{Statistics}
\label{sub:statistics}
\subsubsection{Weighted mean}
Different values $x_i$ with individual errors $s_i$ are given. The weighted (arithmetic) mean and its error is:
\begin{equation}
\label{eq:meanw}
  \bar{x} = \frac{\sum_i \frac{x_i}{s_i^2}}{\sum_i \frac{1}{s_i^2}}, \qquad s_{\bar{x}}^2 = \frac{1}{\sum_i \frac{1}{s_i^2}}
\end{equation}