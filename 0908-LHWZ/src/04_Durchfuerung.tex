\section{Versuchsdurchf�hrung}

\subsection{Bestimmung der Z�hlrohrcharakteristik mit 238U}

\subsection{Bestimmung der Halbwertszeit von 147Sm}

\subsection{Bestimmung der Halbwertszeit von 40K}
Im letzten Versuchsteil wird $\beta$-Strahlung untersucht.
Dazu wird eine Aluminiumschale mit ca. 2\,g KCl bef�llt und in das Z�hlrohr
eingesetzt. Die Z�hlrohrspannung wird in 100\,V-Schritten zwischen 2500\,V und
4000\,V variiert.
Anschlie�end wird eine Aktivit�tsmessung bei 3200\,V Z�hlrohrspannung an der
selben Probe durchgef�hrt. Ein Sch�tzwert f�r die Aktivit�t bei dieser Spannung
ist aus obiger Messung schon vorhanden. Mit diesem Sch�tzwert l�sst sich die
ben�tigte Messzeit berechnen, um einen relativen statistischen Fehler von
2\,\% auf die Aktivit�t zu erhalten. Die Messzeit betr�gt 420\,s.
