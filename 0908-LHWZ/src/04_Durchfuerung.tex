\section{Versuchsdurchf�hrung}

\subsection{Bestimmung der Z�hlrohrcharakteristik mit ${}^{238}\!\text{U}$}
Vor Beginn der Messung wird der Wert der Schwellenspannung des
Einkanalanalysators auf 0.75 eingestellt, so dass gerade kein Rauschen mehr
sichtbar ist. Ein Uranpr�parat wird in das Z�hlrohr eingelegt und die
Charakteristik des Z�hlrohrs bestimmt: F�r Spannungen zwischen 1000\,V und
4000\,V (Schrittweite 100\,V) werden jeweils 50\,s lang die Z�hlereignisse
detektiert. Vor jeder Messung wird eine 15\,s lange Pause gemacht, so dass sich
die Spannung stabilisieren kann. Diese Einstellung wird f�r alle folgenden
Messungen beibehalten.\\
Anschlie�end wird die gleiche Messung ohne das Uranpr�parat erneut durchgef�hrt,
um die Untergrundstrahlung zu bestimmen.

\subsection{Bestimmung der Halbwertszeit von ${}^{147}\!\text{Sm}$}
In diesem Abschnitt wird die energiereiche $\alpha$-Strahlung von Samariumoxid
untersucht. Dazu wird eine mit SmO gef�llte Aluminiumschale in das Z�hlrohr
gestellt und die Z�hlraten f�r Spannungen zwischen 1000\,V und 2200\,V
(Schrittweite 100\,V) jeweils 50\,s lang gemessen. Um die Z�hlrate auf XXX
Prozent genau zu ermitteln, wird bei 1600\,V eine 1200\,s lange Messung
durchgef�hrt.\\
Mit zwei weitere Aluminiumschalen mit kleineren Fl�chen wird ebenfalls die
Aktivit�t 



\subsection{Bestimmung der Halbwertszeit von ${}^{40}\!\text{K}$}
Im letzten Versuchsteil wird die $\beta^-$-Strahlung von Kalium untersucht.
Dazu wird eine Aluminiumschale mit ca. 2\,g KCl bef�llt und in das Z�hlrohr
eingesetzt. Die Z�hlrohrspannung wird in 100\,V-Schritten von 2500\,V auf
4000\,V gesteigert. F�r jede Spannungseinstellung werden 50\,s lang die Zerf�lle
gez�hlt.\\
Anschlie�end wird eine Aktivit�tsmessung bei 3200\,V Z�hlrohrspannung an der
selben Probe durchgef�hrt. Ein Sch�tzwert f�r die Z�hlrate bei dieser Spannung
ist aus obiger Messung schon vorhanden. Mit diesem Sch�tzwert l�sst sich die
ben�tigte Messzeit berechnen [{Formel!!}], um einen relativen statistischen
Fehler von weniger als 2\,\% auf die Aktivit�t zu erhalten. Sie betr�gt
420\,s.
Anschlie�end wird schrittweise die Masse des KCl in der Aluminiumschale
verringert. F�r jede Masse wird 50\,s lang die Anzahl der Z�hlereignisse
registriert und dann jeweils die Messdauer bestimmt, die f�r den Fehler von
2\,\% notwendig ist. Die Messdauer nimmt zu bis auf 780\,s.\\
Schlie�lich wird mit dem leeren Alubeh�lter der Untergrund gemessen. Um einen
nichtbeitragenden Fehler zu erhalten, w�re eine Messdauer von XXXX\,s n�tig [Formel!!!]. Aus Zeitmangel wird
die Dauer der Messung auf 10\,h beschr�nkt.
