\section{Versuchsziel}
Im Versuch wird mit modernen Messmethoden ein Experiment durchgeführt,
das eine Voraussage der speziellen Relativitätstheorie überprüft:
Der \emph{"Mitführungskoeffizient"} von Quarz,
der aus dem Gesetz der relativistischen Addition von Geschwindigkeiten folgt, wird bestimmt.
Dazu wird ein Helium-Neon-Laser mit offenem Resonator verwendet und
eine rotierende Quarzscheibe in den Strahlengang eingesetzt,
die abhängig von ihrer Rotationsgeschwindigkeit die effektive Resonatorlänge ändert,
damit das Laserlicht in zwei Frequenzkomponenten aufspaltet und
so eine messbare Schwebung der Laserlichtintensität verursacht.
