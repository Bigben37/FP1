\subsection{Versuchsdurchführung}

Am Aufbau werden zwei Messreihen durchgeführt:
Zuerst wird das Signal am Oszilloskop in Abhängigkeit der Entfernung Lichtleiter-Nadel untersucht.
Die Entfernung (Anzeige auf der Skala) wird zwischen 0\,mm und 10\,mm in 1\,mm-Schritten variiert.
Es liegt immer die maximal mögliche Treibspannung (50\,V) an.
Bei jeder Messung wird der Offset des Signals so eingestellt,
dass es gut am Oszilloskop ablesbar ist und die Daten vom Oszilloskop gespeichert.
Zusätzlich wird mit einer Schiebelehre der Offset zwischen der Mitte des Lichtleiters und der Nadel bestimmt,
wenn die Entfernungsskala auf 0\,mm eingestellt ist.\\
Bei der zweiten Messreihe wird der Abstand der Nadel konstant auf 3.15\,mm (+Offset)
gehalten und der Wert der Treibspannung in 3\,V-Schritten zwischen 20\,V und 50\,V variiert.
Auch hier wird für jede Messung am Oszilloskop eine geeignete Einstellung gesucht und die Daten gespeichert.