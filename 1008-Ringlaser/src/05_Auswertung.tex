\section{Auswertung}
\subsection{Variabler Auftreffpunkt \texorpdfstring{$x_0$}{x0}}
Für jede Messreihe (konstante Periodendauer $T$) werden die Differenzenfrequenzen $\Delta \nu$ aus der Anzahl der Maxima $N$ und dem zeitlichen 
Abstand $\Delta t$ berechnet.
\begin{equation}
  \label{eq:delta_nu}
  \Delta \nu = \frac{N}{\Delta t}, \qquad s_{\Delta \nu} = \Delta \nu \cdot \frac{s_{\Delta t}}{\Delta t} = \frac{N \cdot s_{\Delta t}}{\Delta t^2}
\end{equation}
Der Fehler der Zeitdifferenz $\Delta t$ wurde auf $s_{\Delta t} = 10$\,\textmu s abgeschätzt.\footnote{Dies ist das 5-fache des protokollierte Fehlers, da 
zusätzliche Fehler (z.B. Überlagerung von verschiedenen Lasermoden) auftreten, die nicht vom Ablesen der Maxima herrühren.} \\
Aus \autoref{eq:alpha:exp} folgt:
\begin{equation}
  \label{eq:x0:m}
  x_0(\Delta \nu) = \frac{\lambda \cdot L}{2 \cdot n \cdot d \cdot \omega \cdot \alpha} \cdot \Delta \nu =: m \cdot \Delta \nu
\end{equation}
Allerdings unterscheidet sich der gemessene Auftreffpunkt $x_0'$ aufgrund des Messaufbaus um einem Skalenoffset $x_m$, 
welcher noch berücksichtigt werden muss.
\begin{equation}
  x_0' = x_0 - x_m
\end{equation}
Die Daten können also mit folgender Formel beschrieben werden:
\begin{equation}
  x_0(\Delta \nu) = m \cdot \Delta \nu + x_m
\end{equation}
Die gemessenen Werte werden nun in ein $x_0'$-$\Delta \nu$-Diagramm eingetragen und mit der obigen Formel gefittet 
(\autoref{img:fit:x0:30ms}, \autoref{img:fit:x0:45ms} und \autoref{img:fit:x0:60ms}). Durch diese Wahl der Koordinatenachsen kann der Offset $x_m$ 
nun direkt abgelesen werden.
Der geschätzte Fehler auf $x_0'$ beträgt $s_{x_0'} = 0.05$\,mm. Die Ergebnisse sind in \autoref{tab:fit:x0} dargestellt.

\begin{figure}[H]
\begin{center}
  \includegraphics[width=\textwidth]{../img/fit_T_30ms.pdf}
  \caption{Linearer Fit von $x_0'$ bei $T = 30$\,ms.}.
  \label{img:fit:x0:30ms}
\end{center}
\end{figure}

\begin{figure}[H]
\begin{center}
  \includegraphics[width=\textwidth]{../img/fit_T_45ms.pdf}
  \caption{Linearer Fit von $x_0'$ bei $T = 45$\,ms.}.
  \label{img:fit:x0:45ms}
\end{center}
\end{figure}

\begin{figure}[H]
\begin{center}
  \includegraphics[width=\textwidth]{../img/fit_T_60ms.pdf}
  \caption{Linearer Fit von $x_0'$ bei $T = 60$\,ms.}.
  \label{img:fit:x0:60ms}
\end{center}
\end{figure}

Der $\chi^2$-Wert der dritten Messreihe (\autoref{img:fit:x0:60ms}) ist zu groß und liegt nicht mehr in den üblichen Konfidenzintervallen. Eine 
Erklärung könnten die unregelmäßigen, nicht ganz sinusförmige Signale sein, die zu hohen Schwankungen der Messwerte führten. Jedoch lässt sich 
trotzdem noch eine Gerade als Trend erkennen, weshalb die Fitparameter in der weiteren Auswertung trotzdem mitbenutzt wurden.

\begin{table}[H]
\caption{Fitergebnisse von $x_0'(\Delta \nu)$ bei festen Periodendauern $T$.}
\begin{center}
\begin{tabular}{|c|c|c|c|c|}
  \hline
  $T$ / ms & $x_m$ / mm & $s_{x_m}$ / mm & $m$ / (mm / kHz) & $s_m$ / (mm / kHz) \\ \hline
  30 & 46.787 & 0.032 & 0.3248 & 0.0018 \\ \hline
  45 & 46.795 & 0.028 & 0.4905 & 0.0022 \\ \hline
  60 & 47.324 & 0.045 & 0.6597 & 0.0040 \\ \hline
\end{tabular}
\end{center}
\label{tab:fit:x0}
\end{table}


Der gewichtete Mittelwert aus den Offsets $x_m$ für verschiedene Periodendauern $T$ liefert:
\begin{equation}
  \label{eq:xm}
  \bar{x}_{m} = (46.886 \pm 0.019)\,\text{mm}
\end{equation}
Aus den einzelnen Steigungen $m$ lässt sich nun der Mitführungskoeffizient $\alpha$ nach \autoref{eq:x0:m} bestimmen. Die Werte der Konstanten des 
Versuchaufbaus sind in \autoref{sec:aufbau} aufgelistet.
\begin{equation}
  \alpha = \frac{\lambda \cdot L}{2 \cdot n \cdot d} \cdot \frac{1}{\omega \cdot m}, \qquad
  s_{\alpha} = \alpha \cdot \sqrt{\left(\frac{s_m}{m}\right)^2 + \left(\frac{s_\omega}{\omega}\right)^2}
\end{equation}
Die Kreisfrequenz $\omega$ lässt sich aus der Periodendauer $T$ folgendermaßen bestimmen:
\begin{equation}
  \label{eq:omega}
  \omega = \frac{2 \pi}{T}, \qquad s_{\omega} = \omega \cdot \frac{s_T}{T}
\end{equation}
wobei der Fehler auf der Periodendauer $T$ auf $s_T = 0.2$\,ms geschätzt wurde.
Die berechneten Mitführungskoeffizienten sind in \autoref{tab:x0:alpha} aufgelistet.
\begin{table}[H]
\caption{Mitf\"uhrungskoeffizienten bei festen Perioden $T$. }
\begin{center}
\begin{tabular}{|c|c|c|}
  \hline
  $T$ / ms & $\alpha$ & $s_{\alpha}$ \\ \hline
  30 & 0.540 & 0.005 \\ \hline
  45 & 0.537 & 0.003 \\ \hline
  60 & 0.532 & 0.004 \\ \hline
\end{tabular}
\end{center}
\label{tab:x0:alpha}
\end{table}

Da alle Mitführungskoeffizienten innerhalb einer Standardabweichung übereinstimmen, kann der gewichtete Mittelwert $\bar{\alpha}$ gebildet werden.
\begin{equation}
  \label{eq:x0:alpha:avg}
  \bar{\alpha} = 0.536 \pm 0.002
\end{equation}

\subsection{Variable Periodendauer \texorpdfstring{$T$}{T}}
Es werden wieder die einzelnen Differenzenfrequenzen mit \autoref{eq:delta_nu} berechnet. Hier ist der Fehler $s_{\Delta t}$ 
auf die Zeitdifferenz $\Delta t$ entweder 5 oder 10 \textmu s, je nach verwendeter Auflösung des Oszilloskops.\footnote{Auch hier ist der 
verwendete Fehler das 5-fache des angegebenen Fehlers.} \\
Die gemessenen Werte werden in einem $\Delta \nu$-$\omega$-Diagramm dargestellt. 
Dazu werden die einstellten Periodendauern $T$ mit \autoref{eq:omega} in Kreisfrequenzen $\omega$ umgerechnet. \\
Die Daten können wieder mit einer Geraden beschrieben werden:
\begin{equation}
  \label{eq:nu_omega}
  \Delta \nu (\omega) = a + \frac{2 \cdot n \cdot d \cdot x_0 \cdot \alpha}{\lambda \cdot L} \cdot \omega  =: a + b \cdot \omega
\end{equation}
Der Offset $a$ sollte im Idealfall verschwinden, ist er jedoch nicht 0, so gibt er Information über einen eventuellen systematischen Fehler.
Die Fits für die verschiedenen, festen Auftreffpunkte $x_0$ sind in \autoref{img:fit:T:36mm}, \autoref{img:fit:T:40mm}, \autoref{img:fit:T:53mm} 
und \autoref{img:fit:T:57mm} dargestellt. Die Ergebnisse für den Offset $a$ und die Steigung $b$ sind in \autoref{tab:fit:T} aufgelistet.

\begin{figure}[H]
\begin{center}
  \includegraphics[width=\textwidth]{../img/fit_x0_36mm.pdf}
  \caption{Linearer Fit von $\Delta \nu$ bei $x_0' = 36$\,mm.}
  \label{img:fit:T:36mm}
\end{center}
\end{figure}

\begin{figure}[H]
\begin{center}
  \includegraphics[width=\textwidth]{../img/fit_x0_40mm.pdf}
  \caption{Linearer Fit von $\Delta \nu$ bei $x_0' = 40$\,mm.}
  \label{img:fit:T:40mm}
\end{center}
\end{figure}

\begin{figure}[H]
\begin{center}
  \includegraphics[width=\textwidth]{../img/fit_x0_53mm.pdf}
  \caption{Linearer Fit von $\Delta \nu$ bei $x_0' = 53$\,mm.}
  \label{img:fit:T:53mm}
\end{center}
\end{figure}

\begin{figure}[H]
\begin{center}
  \includegraphics[width=\textwidth]{../img/fit_x0_57mm.pdf}
  \caption{Linearer Fit von $\Delta \nu$ bei $x_0' = 57$\,mm.}
  \label{img:fit:T:57mm}
\end{center}
\end{figure}

Auch hier streuen die Werte mehr oder weniger um das theoretische Modell, was auf die gleiche Ursache wie oben zurückzuführen ist.

\begin{table}[H]
\caption{Fitergebnisse von $\Delta \nu(\omega)$ bei festen Auftrittpunkten $x_0'$.}
\begin{center}
\begin{tabular}{|c|c|c|c|c|}
  \hline
  $x_0'$ / mm & $a$ / kHz & $s_{a}$ / kHz & $b$ / (kHz $\cdot$ ms) & $s_b$ / (kHz $\cdot$ ms) \\ \hline
  36 & -0.6 & 0.5 & 164 & 4 \\ \hline
  40 & 1.0 & 0.3 & 93 & 2 \\ \hline
  53 & 0.5 & 0.3 & 88 & 2 \\ \hline
  57 & 0.3 & 0.5 & 147 & 4 \\ \hline
\end{tabular}
\end{center}
\label{tab:fit:T}
\end{table}


Die Offsets $a$ verschwinden alle innerhalb von maximal 4 Standardabweichungen. Aus den Steigungen $b$ lässt sich nun mit \autoref{eq:nu_omega} 
der Mitführungskoeffizient $\alpha$ bestimmen (alle Konstanten aus \autoref{sec:aufbau}):
\begin{equation}
  \alpha = \frac{\lambda \cdot L}{2 \cdot n \cdot d} \cdot \frac{b}{x_0}, \qquad
  s_\alpha = \alpha \cdot \sqrt{ \left( \frac{s_b}{b} \right)^2 + \left( \frac{s_{x_0}}{x_0} \right)^2 }
\end{equation}
Der Auftreffpunkt $x_0$ berechnet sich aus dem gemessenen Auftreffpunkt $s_0'$ (Fehler $s_{x_0'} = 0.05$\,mm) und dem 
Offset des Auftreffpunkts $x_m$, welcher oben (\autoref{eq:xm}) bestimmt wurde.
\begin{equation}
  x_0 = x_0' - x_m, \qquad s_{x_0} = \sqrt{s_{x_0'}^2 + s_{x_m}^2}
\end{equation}
Die verschiedenen Mitführungskoeffizienten sind in \autoref{tab:T:alpha} aufgelistet.
\begin{table}[H]
\caption{Mitf\"uhrungskoeffizienten bei festen Auftreffpunkten $x_0'$. }
\begin{center}
\begin{tabular}{|c|c|c|}
  \hline
  $T$ / ms & $\alpha$ & $s_{\alpha}$ \\ \hline
  36 & 0.554 & 0.014 \\ \hline
  40 & 0.498 & 0.012 \\ \hline
  53 & 0.529 & 0.013 \\ \hline
  57 & 0.532 & 0.016 \\ \hline
\end{tabular}
\end{center}
\label{tab:T:alpha}
\end{table}

Sie stimmen innerhalb von drei Standardabweichungen überein, deshalb kann wieder der gewichtete Mittelwert $\bar{\alpha}$ gebildet werden.
\begin{equation}
  \bar{\alpha} = 0.526 \pm 0.007
\end{equation}

\subsection{Vergleich der verschiedenen Mitführungskoeffizienten}
Der theoretische Mitführungskoeffizient $\alpha_{\text{theo}}$ lässt sich mit \autoref{eq:alpha:theo} mit dem Brechungsindex von Quarzglas $n=1.457$ 
und der Dispersion $\frac{\difd n}{\difd \lambda} = -300\,\text{cm}^{-1}$ bei der Wellenlänge  $\lambda = 632.8$\,nm berechnen.
\begin{equation}
  \alpha_{\text{theo}} \approx 0.542
\end{equation}
Es werden zum Vergleich nochmal die gemittelten Mitführungskoeffizienten bei variablen Periodendauern $\alpha_T$ und bei variablen Auftreffpunkten 
$\alpha_{x_0}$ aufgelistet.
\begin{equation}
\begin{split}
  \alpha_T = 0.526 \pm 0.007 \\
  \alpha_{x_0} = 0.536 \pm 0.002
\end{split}
\end{equation}
Die beiden stimmen innerhalb von zwei Standardabweichungen überein und der gewichtete Mittelwert lautet:
\begin{equation}
  \alpha_{\text{exp}} = 0.535 \pm 0.002
\end{equation}
Der theoretische Wert liegt innerhalb des $4$-$\sigma$-Intervalls dieses Wertes. Die prozentuale Abweichung des gemessenen Wertes beträgt
\begin{equation}
  \frac{0.542 - 0.535}{0.542} \approx 1.3 \%
\end{equation}
was ein akzeptables Ergebnis ist. \\
Eine wahrscheinliche Fehlerquelle ist das Vorliegen von verschiedenen Lasermoden, was zu Schwankungen der Periodendauer des  gemessenen Signals 
geführt hat. Anscheinend wurden durch die Fehlerabschätzung im Protokoll nicht alle Fehlerquellen berücksichtigt, was sich an erhöhten 
$\chi^2$-Werte zeigt.