\section{Versuchsziel}
Ziel des Versuches ist die Untersuchung kernmagnetischer Eigenschaften von Wasserstoff und Fluor:
In einem starken statischen Magnetfeld werden Energieniveaus des Kerns aufgespalten (Zeeman-Effekt)
und dann die feldstärkenabhängige Frequenz bestimmt, bei der Photonen einen Übergang
zwischen den beiden Energieniveaus induzieren und absorbiert werden.
Aus dieser Frequenz können das \emph{gyromagnetische Verhältnis},
der \emph{Kern-g-Faktor} und das \emph{kernmagnetische Moment} bestimmt werden.