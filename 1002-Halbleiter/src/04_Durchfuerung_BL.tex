\subsection{Versuchsdurchführung}

Am Aufbau werden an Germanium und Silizium die gleichen Messungen durchgeführt.
Zu Beginn wird die Nullwinkelstellung des optischen Gitters mit Hilfe des Maximums 0.\,Ordnung gesucht.
Anschließend wird von diesem Maximum ausgehend eine Trans\-missions- und Absorptionsmessung
über den gesamten Winkelbereich durchgeführt.
Außerdem wird mit abgedecktem Halbleiter der Untergrund und ohne Halbleiter das Spektrum der Lampe
mit dem Pyrodetektor vermessen.
Um die Messfehler abzuschätzen, werden an den Maxima von Transmission und Absorption für jeweils 100\,s
halbsekündlich Messwerte aufgenommen.\\
\autoref{tab:einstBL} zeigt die Einstellungen am Lock-in-Verstärker, die für die beiden Messungen verwendet wurden.


\begin{table}[H]
\caption{Einstellungen am Lock-in-Verstärker bei der Bestimmung der Bandlücken.}
\begin{center}
\begin{tabular}{|c|c|c|c|c|c|}
\hline
			&	$I$/mA	&	AC gain pyro	&	AC gain sample	&	DC gain pyro	&	DC gain sample	\\ \hline
Silizium	&	0.75	&					&					&					&					\\ \hline
Germanium	&	15.00	&	300				&	1000			&10					&	5				\\ \hline
\end{tabular}
\end{center}
\label{tab:einstBL}
\end{table}
%TODO Lockin Daten Si