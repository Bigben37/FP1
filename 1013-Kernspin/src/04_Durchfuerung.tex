\section{Versuchsdurchführung}




\subsection*{Charakterisierung des Aufbaus}
Vor Durchführung der Kernspinresonanz-Messung werden
zwei Messreihen zur Charakterisierung des Aufbaus durchgeführt:\\
Mit der Hall-Sonde wird der gesamte Bereich der Probenhalterung in senkrechter
Richtung abgefahren und in 5\,mm-Schritten die Stärke des Magnetfelds bestimmt.\\
Dann wird bei fester Position der Hall-Sonde (20\,mm unter der Oberkante der Öffnung) die Abhängigkeit
der Magnetfeldstärke vom Spulenstrom untersucht,
um später schnell gewünschte Feldstärken einstellen zu können.

\subsection*{Messung der Kernspinresonanz an Wasser, Teflon und Glykol}
Die Messung der Kernspinresonanz an Wasser wird für feste Frequenzen des HF-Wechselfeldes
(zwischen 16\,MHz und 19\,MHz in 0.5\,MHz-Schritten)
durchgeführt und für jede Frequenz der Strom für das statische Magnetfeld am Netzteil so eingestellt,
dass am Oszilloskop die Minima der Absorption gleich weit voneinander entfernt sind.
Für jeden eingestellten Strom wird durch eine Messung mit der Hall-Sonde die Magnetfeldstärke gemessen.\\
Danach wird mit dem selben Messprinzip je ein Messpunkt bei 17.5\,MHz für Teflon und für Glykol aufgenommen.
