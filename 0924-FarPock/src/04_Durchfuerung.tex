\section{Versuchsdurchführung}
\subsection{Teil I: Pockels-Effekt}
\subsubsection{Bestimmung der Halbwellenspannung mit einer Sägezahnspannung}
Um eine grobe Abschätzung der Halbwellenspannung zu bekommen,
wird im ersten Versuchsteil an die Pockelszelle das Sägezahnsignal angelegt.
Gleichzeitig wird mit dem Oszilloskop das Spannungssignal nach dem Verstärker der Photodiode gemessen.
Durch Drehen des Analysators wird dieses Signal so eingestellt,
dass die Amplitude maximal ist, aber kein \emph{Clipping} auftritt.
Die beiden Signale werden mit dem Computer aufgezeichnet.
Anschließend wird der Analysator für zwei weitere Messungen jeweils ein wenig weiter verdreht,
um eine eventuelle Amplitudenabhängigkeit der Halbwellenspannung zu identifizieren. 


\subsubsection{Bestimmung der Halbwellenspannung mit einer Gleichspannung}

\subsection{Teil II: Faraday-Effekt}
-erwähne Messreihenfolge --> Aufspüren systematischer Fehler